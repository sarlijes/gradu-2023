\documentclass[utf8]{gradu3}

\usepackage[utf8]{inputenc}
\usepackage{graphicx} % kuvien mukaan ottamista varten
\usepackage{amsmath} % hyödyllinen jos tekstisi sisältää matikkaa,
                     % ei pakollinen
\usepackage{booktabs} % hyvä kauniiden taulukoiden tekemiseen
\usepackage{comment}
\usepackage{tikz}
\usepackage{tabularray}
\usepackage{xcolor}

\usepackage{tabularx}
\usepackage{ragged2e}
\usepackage{adjustbox}
\usepackage{lipsum}
\usepackage{longtable}

% HUOM! Tämän tulee olla viimeinen \usepackage koko dokumentissa!
\usepackage[bookmarksopen,bookmarksnumbered,linktocpage]{hyperref}

\addbibresource{gradulahteet.bib} % Lähdetietokannan tiedostonimi


\begin{document}

\title{Systemaattinen kirjallisuuskatsaus ohjelmistokehitysalan perehdyttämiskäytännöistä (v. 0.3.2)}

\translatedtitle{Systematic literature review of onboarding practices in software engineering}
\studyline{Ohjelmisto- ja tietoliikennetekniikan opintosuunta}
\avainsanat{TODO}
\keywords{TODO}

\tiivistelma{TODO}
\abstract{TODO}

\author{Jessica Sarlin}
\contactinformation{\texttt{jessica.sarlin@gmail.com}}

\supervisor{Antti-Jussi Lakanen}

\maketitle

\mainmatter

\chapter{Johdanto}

% Miksi työ ohjelmistoalalla on haastavaa? (asiakkaiden lukuisat toiveet, rajallinen käytettävissä oleva aika, haastavaa rakentaa ohjelmistoja, jotka on helppo käyttää, ylläpitää ja laajentaa, yksittäiseen työtehtävään kuluvaa aikaa vaikea arvioida etukäteen)

% Kun maailma digitalisoituu ja kaikkialla on koodia, tarvitaan jatkuvasti vahvempaa ja parempaa osaamista. Ihan ok ei riitä, jos ja kun maailma ja kaikki yhteiskunnan kriittiset asiat pyörii koodilla!

Hyvä perehdytys työsuhteen alkaessa on tärkeää sekä työntekijöille että työnantajille. Huolellisella perehdytyksellä voidaan varmistaa, että uudet työntekijät saavat työssään tarvitsemansa tiedot ja taidot. Se on tärkeä osa työntekijöiden integroitumista työyhteisöön ja auttaa heitä omaksumaan työkulttuuriin ja -käytäntöihin liittyviä asioita. Hyvä perehdytys on tärkeää heidän onnistumiselleen, sitoutumiselleen ja tyytyväisyydelleen työssään, ja sen avulla voidaan myös vähentää virheitä ja parantaa työskentelytehokkuutta. Erityisen tärkeää perehdytys on työuran alussa.

Ohjelmistokehityksen alalla on paljon tarvetta kokeneille osaajille eli senioreille. Jokainen alalle tulija kuitenkin aloittaa juniorina, josta vähitellen kehitytään kokeneeksi osaajaksi - esimerkiksi \textcite{bologa-lupu-2014} arvioivat tämän voivan kestää jopa kaksi vuotta. Ohjelmistokehitystyön luonne abstraktina ongelmanratkaisuna tuo omat haasteensa perehdyttämiselle. On tärkeää, että yksilöiden osaaminen kehittyy, jotta alalla on jatkossakin osaavia asiantuntijoita.

Tämän tutkimuksen tavoitteena on selvittää ohjelmistokehitysorganisaatioiden käyttämiä perehdyttämiskäytäntöjä ja luoda niistä synteesiä, jotta niitä voitaisiin helpommin hyödyntää. Tutkimusmenetelmänä on systemaattinen kirjallisuuskatsaus.

\textbf{TODO lisää akateeminen motivaatio}

\textbf{TODO kerrottava miksi tehtiin, ja sekä akateemista että käytännön merkitystä}

\chapter{Teoriaosa}

Tässä luvussa käsitellään perehdyttämisen haasteita ohjelmistoalalla sekä perehdyttämisen käsitettä, tärkeyttä ja hyödyjä. Tulokkaiden perehdytyskäytäntöjä jäsennetään sosialisaatioresurssien teorian \parencite{saks-gruman-2012} mukaisesti. Tulokas (engl. \textit{newcomer}) on kirjallisuudessa usein käytetty termi %
\parencites%
    [mm.][]{matturro-ym-2017}%
    {begel-simon-2008-all-over-again}%
    {steinmacher-ym-2015}%
    {britto-ym-2019}%
\relax.
%
Tässä katsauksessa sillä tarkoitetaan uutta työntekijää. Termi ei erottele sitä, onko työntekijä vastavalmistunut, juniori, kokenut vai kenties siirtynyt organisaation sisällä tehtävästä toiseen.

\section{Perehdyttämisen haasteet ohjelmistoalalla}

Miksi tulokkaiden perehdyttäminen on ohjelmistoalalla erityisen haastavaa? Esimerkiksi \textcite{bologa-lupu-2014} toteavat, että kehittyminen juniorista osaajaksi kestää tyypillisesti noin kaksi vuotta. \textcite{begel-simon-2008-all-over-again} toteavat ohjelmistokehittäjien kohtaavan haastavan vaiheen ensimmäisissä työpaikoissaan vastavalmistuneina - ohjelmistokehityksen opiskeleminen kun antaa vain perustaidot alalla työskentelyyn. \textcite{begel-simon-2008-all-over-again} toteavat, että työelämässä ohjelmistokehittäjien tulee paitsi osata ohjelmoida, suunnittella ja testata ohjelmia, myös pystyä muokkaamaan, debuggaamaan, ja luomaan uutta koodia yhteistoiminnassa kollegoidensa kanssa ja oppimaan paljon uusia taitoja, tekniikoita ja menetelmiä. Ohjelmistokehittäjän työ tosiaan vaatii monia taitoja. \textcite{swebok} ovat toimittaneet SWEBOK (Software Engineering Body of Knowledge) -julkaisun, joka kuvaa ohjelmistokehityksen yleiset osaamisvaatimukset. Ne on jaettu 15 osa-alueeseen kuten ohjelmistokehityksen prosessit, vaatimusmäärittely, suunnittelu, testaus, ylläpito, konfiguraatioidenhallinta, ammattikäytännöt ja laadunvarmistus \parencite{swebok}. Myös \textcite{gregory-ym-2020} toteavat, että ohjelmistokehittäjältä vaaditaan korkeatasoista teknistä taitoa, laajaa tietämystä, vahvaa kokemusta ja kohdealuetuntemusta. Näin laajan osaamisen saavuttaminen vaatii aikaa ja harjoitusta.

Lisäksi voidaan todeta, että ohjelmistokehittäjillä voi olla vaihtelevia koulutustaustoja. Osa on itseoppineita ja osa valmistunut erilaisista oppilaitoksista kuten ammattikouluista, ammattikorkeakouluista tai yliopistoista. Myös bootcamp-tyyppinen lyhyt kurssitus tuottaa ohjelmistoalan osaajia. 

Nykyisin ohjelmistotekniikkaa käytetään mitä erilaisimmissa yhteyksissä. Esimerkiksi yrityksen toiminnanohjausjärjestelmän, mobiilipelin ja auton ajotietokoneen ohjelmistot poikkeavat toisistaan paljon. Näin erilaisten ohjelmistojen kehittämiseenkin toki pätevät erilaiset lainalaisuudet. Tämän vuoksi ohjelmistokehitystä tehdään eri organisaatioissa eri painotuksin, mikä asettaa vaatimuksia organisaatioiden perehdytyskäytännöille. Mikään oppilaitos ei voi tuottaa täysin valmiita osaajia, vaan organisaatioiden perehdytyskäytännöillä on tässä tärkeä rooli.

\section{Tulokkaiden haasteet ohjelmistoalalla}
\label{luku-tulokkaiden-haasteet}

Ohjelmistokehittäjät kohtaavat ensimmäisissä työpaikoissaan monenlaisia haasteita. Esimerkiksi \textcite{begel-simon-2008} raportoivat haasteita viidellä osa-alueella: kommunikaatio, yhteistoiminta, tekninen taito, kognitio ja orientoituminen. He toteavat, että kommunikaation haasteet liittyvät erityisesti siihen, että tulokkaat eivät pyydä apua tai kysy tarkentavia kysymyksiä kyllin aikaisin. Myös \textcite{radermacher-ym-2015} toteavat vastavalmistuneiden ohjelmistokehittäjien jättävän usein kysymyksiä esittämättä välttääkseen näyttämästä osaamattomuuttaan. \textcite{begel-simon-2008} jatkavat, että vasta-alkajilla on vaikeuksia hahmottaa kuhunkin tilanteeseen sopivaa detaljien tasoa: toisinaan yksityiskohtaisen tiedon puute aiheuttaa väärinkäsityksiä ja toisinaan taas yksityiskohtiin takertuminen estää aiheen laajempaa käsittelyä. Myös \textcite{radermacher-ym-2015} toteavat vasta-alkajien kokevan usein haasteita niin kirjallisessa, suullisessa kuin asiakkaidenkin kanssa tapahtuvassa kommunikoimisessa. \textcite{moe-ym-2020} taas mainitsevat globaalisti hajautetun ohjelmistokehityksen osalta myös haasteet viestintätyökalujen toimivuudessa.

\textcite{begel-simon-2008} toteavat, että tulokkaiden yhteistoimintaan liittyvät ongelmat liittyvät työskentelyyn suurissa tiimeissä tai useiden tiimien kanssa. Myös \textcite{britto-ym-2019} havaitsivat, että tutkitut ohjelmistokehittäjät työskentelivät jatkuvasti vaihtuvissa tiimeissä, jolloin ryhmäoppiminen hankaloitui, mikä vaikutti ohjelmistokehittäjien tehokkuuteen. \textcite{begel-simon-2008} kertovat, että tulokkaiden tulee oppia tiettyä assertiivisuutta, jotta he pystyvät sekä rajaamaan omaa työtään (sen sijaan että ottaisivat kollegoidensa pyynnöstä uusia työtehtäviä välittömästi työn alle) että toimimaan aktiivisesti saadakseen tuottamansa ohjelmistokoodin katselmoituksi (sen sijaan että katselmointi jäisi tekemättä kollegoiden muiden kiireiden vuoksi).

Riittämättömät tekniset taidot taas aiheuttavat ongelmia versionhallinnan käyttämisessä, ohjelmistojen testaamisessa ja vaikeuksia laajassa ohjelmistokoodissa navigoimisessa \parencite{begel-simon-2008}. Haasteet versionhallinnan käyttämisessä mainitsevat myös \textcite{radermacher-ym-2015}.

Kognitioon liittyy haasteita muistiinpanotekniikoissa sekä siinä, että työnopetus tapahtuu pienissä, strukturoimattomissa hetkissä muun työn ohella, jolloin osaaminen rakentuu sattumanvaraisesti ja vailla jäsennystä. Ohjelmistokehittäjien on myös vaikea tunnistaa, milloin he ovat jääneet jumiin, eivätkä pysty jatkamaan tehtäväänsä ilman apua. \parencite{begel-simon-2008}. Muistiinpanojen ja muiden dokumenttien laatimisen haasteet mainitsevat myös \textcite{radermacher-ym-2015}. \textcite{britto-ym-2019} taas toteavat, että tulokkaiden ottaminen jo varhaisessa vaiheessa mukaan suuriin, monimutkaisiin ja globaalisti hajautettuihin työtehtäviin näyttää vaikeuttavan perehdyttämisprosessia.

Työtehtäviin orientoitumista voi vaikeuttaa vähäinen tai heikko dokumentaatio, massiivinen olemassaolevan lähdekoodin määrä ja jopa se, ettei uusi työntekijä tunne työryhmänsä jäseniä lainkaan \parencite{begel-simon-2008}. Yleisesti näitä vaikeuksia näyttää siis yhdistävän liian vähäinen tiedon määrä. \textcite{begel-simon-2008} kuvaavat esimerkiksi tilannetta, jossa eräs vasta-alkaja käytti paljon aikaa lukiakseen ylätason asiakirjoja etsiessään tietoa siitä, mitä hänen työryhmässään oikeastaan edes tehdään. \textcite{britto-ym-2019} toteavat myös, että dokumentaation puutteet sekä olemassaolevan lähdekoodin määrä ja kompleksisuus vaikeuttavat tulokkaiden työskentelyä.

\textcite{begel-simon-2008} esittävät tutkimustulostensa perusteella kehityskohteita ohjelmistokehittäjiä kouluttaville tahoille. Mielestäni näyttää kuitenkin siltä, monet heidän havaitsemansa ohjelmistokehittäjien haasteet liittyvät itse asiassa enemmän työyhteisön rakenteisiin ja perehdyttämiskäytäntöihin, eivät varsinaisesti vasta-alkajien osaamiseen. Vain yksi viidestä alueesta, jolla \textcite{begel-simon-2008} kuvaavat haasteita olevan, liittyy työn tekniseen tekemiseen. Muut aihealueet ovat kommunikaatio, yhteistoiminta, kognitio ja orientoituminen.

%Suora lainaus: "Another solution is to identify the areas within your own companies where newly hired, recent graduates struggle and to develop an orientation program that is specifically designed to tackle those problems, whether they are related to maintaining professional conduct or developing training material for the different software tools used at the company so that new employees have a quick reference while first learning to work with a new or different tool." \textcite{radermacher-ym-2015}

\section{Organisatorisesta sosialisaatiosta perehdyttämiseen}

Jotta perehdyttämistä voidaan ymmärtää, hahmottaa ja tutkia, on aluksi tarpeen käsitellä sitä, miten perehdyttämisen käsitettä on kirjallisuudessa pyritty määrittelemään.

\textcite{wanberg-2012} toteaa, että perehdyttämisen yläkäsite on organisatorinen sosialisaatio (eng. \textit{organizational socialization}). Sillä \textcite{wanberg-2012} tarkoittaa organisaatioon hiljattain liittyneessä ihmisessä tapahtuvaa prosessia, jossa hän hankkii uuteen työtehtävään sopeutumiseen tarvittavat tiedot, taidot, asenteet ja toimintatavat. Myös \textcite{chao-2012} toteaa, että organisatorinen sosiaalistuminen on oppimis- ja sopeutumisprosessi, jonka avulla yksilö voi omaksua roolin, joka vastaa sekä organisaation että yksilön itsensä tarpeita. \textcite{chao-2012} toteaa organisatorisen sosialisaation käsitteen kattavan sekä organisaation että yksilön pyrkimykset työhön sopeutumiseen.

\textcite{saks-gruman-2012} taas määrittelevät organisatorisen sosialisaation käytännöiksi nimenomaan organisaation aloitteesta tapahtuvat ohjelmat, tapahtumat ja kokonaisuudet, joiden tavoitteena on helpottaa uusien tulokkaiden oppimista ja sopeutumista työhön, työryhmään ja rooliin, varmistaen näin että tulokkaista tulee organisaation tehokkaita jäseniä. \textcite{saks-gruman-2012} tekevät vielä erikseen eron organisaation aloitteesta tapahtuviin toimiin, erottaen ne työntekijän itsensä aloitteesta tehtäviin.

\textcite{wanberg-2012} toteaa, että perehdyttäminen taas on organisatorista sosiaalistumista kapeampi käsite. Hänen mukaansa sosiaalistumisen prosessi voi toki sisältää perehdyttämistä, mutta se kattaa myös laajempia tiedonhaun, oppimisen ja sopeutumisen osaprosesseja. Myös \textcite{klein-polin-2012} korostavat, että sosialisaatio on prosessi, joka tapahtuu yksilön sisällä. He toteavat, että perehdyttämisellä tarkoitetaan niin muodollisia kuin epämuodollisiakin käytäntöjä, toimintatapoja ja toimia, joita organisaatio tai sen edustajat käyttävät helpottaakseen uusien tulokkaiden sopeutumista.

%\textcite{chao-2012} toteaa, että organisatorisen sosialisaation käsitettä on käytetty perehdyttämisen rinnalla. Hän kuitenkin toteaa, että perehdyttämisen käsitettä näytetään käytettävän vain uusien työntekijöiden kohdalla eli ei silloin, kuin työntekijä vaihtaa tehtäviä organisaation sisällä. 

\textcite{klein-polin-2012} määrittelevät perehdyttämiseksi joukon käytäntöjä, toimintatapoja ja menettelyjä, joita esihenkilöt ja HR-osastot käyttävät jäsentääkseen uusien työntekijöiden ensimmäisiä kokemuksia ja helpottaakseen siten uusien työntekijöiden sosiaalistumista.

Myös tässä tutkimuksessa perehdyttämisen käsite ymmärretään nimenomaan organisaation aloitteesta tapahtuvaksi toiminnaksi. Uuden työntekijän oma toiminta vaikuttaa toki organisatoriseen sosiaalistumiseen, mutta tässä tutkimuksessa tarkastelun kohteena ovat nimenomaan organisaatiosta lähtevät käytännöt.

\section{Perehdyttämisen hyödyt}
\label{luku-perehdyttämisen-hyödyt}

Miksi uusien työntekijöiden perehdyttäminen sitten on tärkeää? \textcite{saks-gruman-2012} toteavat, että uudessa työpaikassa aloittavat työntekijät kokevat usein epävarmuuden ja vierauden tunteita liittyen omaan rooliinsa, suoriutumiskykyynsä ja organisaation toimintatapoihin. Uuden työn aloittamiseen ei siis liity ainoastaan työn itsensä suorittamiseen liittyvää epävarmuutta, vaan moniuloitteisia sosiaalisia ulottuvuuksia \parencite{saks-gruman-2012}. 

\textcite{wanberg-2012} toteaa, että onnistunut organisatorinen sosiaalisaatio voi johtaa lisääntyneeseen työntekijöiden tyytyväisyyteen, sitoutuneisuuteen, työssä pysymiseen ja hyvään suoriutumiseen. Myös \textcite{bauer-ym-2007} toteavat, että työntekijöiden sopeutumisen osatekijät (selkeä käsitys omasta roolista, vahva minäpystyvyys ja sosiaalinen hyväksyntä) olisi yhteydessä työssä koettuun tyytyväisyydeen, työhön sitoutumiseen, hyviin työsuorituksiin ja vähäiseen vaihtuvuuteen. 

\textcite{saks-gruman-2012} toteavat, että organitorisen sosialisaation tuloksia tarkasteltaessa voidaan tehdä ero lähitulosten (engl. \textit{distal outcomes}) ja kaukotulosten (eng. \textit{proximal outcomes}) välille. Lähituloksiin on myös viitattu uuden työntekijän sopeutumisena ja esimerkeiksi mainitaan työntekijän kokema roolien selkeys, oppiminen ja minäpystyvyys. Kaukotuloksilla taas tarkoitetaan perinteisempiä sosiaalistumisen tuloksia kuten työtyytyväisyyttä ja organisaatioon sitoutumista. Sosialisaatiokäytäntöjen hyödyntämisen katsotaan johtavan lähituloksiin, joiden puolestaan katsotaan johtavan kaukotuloksiin. \parencite{saks-gruman-2012}.

Vaikuttaa siis siltä, että uusien työntekijöiden perehdyttämiseen panostamalla voidaan edistää työntekijöiden organisatorisen sosialisaation onnistumista. Hyvä perehdytys varmistaa, että uudet tulokkaat saavuttavat työssään tarvittavat tiedot ja taidot sekä onnistuakseen tehtävissään että sitoutuakseen työpaikkaansa.

Miten perehdyttämisen käytännön toteutusta eli perehdyttämiskäytäntöjä sitten voidaan jäsentää? Kirjallisuudessa aihetta on lähestytty eri näkökulmista. \textcite{saks-gruman-2012} jäsentävät organisatorisen sosialisaation kirjallisuudessa esiteltyjä käytäntöjä jakaen ne viiteen ryhmään, minkä pohjalta he esittelevät oman sosialisaatioresurssien teoriansa. Seuraavissa luvuissa tutustutaan näihin viiteen ryhmään ja kyseiseen teoriaan.

\section{Käytäntöjen jaottelua}

\textcite{saks-gruman-2012} siis jakavat organisatorisen sosialisaation käytännöt viiteen ryhmään: orientaatio-ohjelmat, työnopetusohjelmat, sosialisaatiotaktiikat, työn ominaispiirteet ja sosialisaatioagentit.

Näistä orientaatio-ohjelmalla (engl. \textit{orientation programs}) tarkoitetaan työntekijän työtehtävässä aloittamisen jälkeen alkavaa lyhyttä vaihetta, jonka aikana työntekijä saa perustiedot uudesta tehtävästään ja työyhteisöstään. Orientaatio-ohjelman katsotaan olevan siis lähinnä lyhyt alkukatsaus uuteen työhön, joissa usein käsitellään käytännön asioita liittyen esimerkiksi työturvallisuuteen, työehtoihin, organisaation perustietoihin ja HR-käytäntöihin. Orientaatio-ohjelmissa uudet työntekijät saavat tietoa, joka on relevanttia kaikille uusille työntekijöille riippumatta heidän työtehtävistään. \parencite{saks-gruman-2012}.

Työnopetusohjelmat (engl. \textit{training programs}) taas keskittyvät opettamaan uusille työntekijöille ne tiedot ja taidot, jotka tarvitaan juuri heidän uuden tehtävänkuvansa mukaisen työn tekemiseen \parencite{saks-gruman-2012}.

Sosialisaatiotaktiikat (engl. \textit{socialization tactics}) taas ovat menettelyjä, joilla usein esihenkilöt pyrkivät sosiaalistamaan työntekijöitä osaksi työyhteisöä. Näitä voivat olla esimerkiksi yhteistoiminta muiden uusien työntekijöiden kanssa tai työnopetuksen järjestäminen tietyssä järjestyksessä. \parencite{saks-gruman-2012}. Näyttää siis siltä, että työnopetusohjelmien keskittyessä työn sisältöihin, sosialisaatiotaktiikat kuvaavat pikemminkin sitä, \textit{miten} työnopetus järjestetään.

Neljäs organisatorisen sosialisaation käytäntöjen ryhmä liittyy työn ominaispiirteisiin (engl. \textit{job characteristics}). \textcite{saks-gruman-2012} viittaavat Katziin, joka mukaan näitä ovat taitojen monipuolisuus, tehtävän identiteetti, tehtävän merkitys, autonomia ja työn palaute. Näiden merkityksellisyys ja yhteys työntekijän asenteisiin ja käyttäytymiseen vaihtelee työsuhteen aikana. Ajan kuluessa siis eri ominaisuudet ja työntekijän reaktiot niihin saavat erilaisia merkityksiä. (\textcite{katz-1980}, \textcite{saks-gruman-2012} mukaan.)

Viimeinen ryhmä taas liittyy sosialisaatioagentteihin (engl. \textit{socialization agents}), jotka ovat organisaation sisällä olevia toimijoita, jotka auttavat uusia työntekijöitä oppimaan työhön liittyviä tietoja, taitoja, rooleja ja identiteettejä \parencite{saks-gruman-2012}.

\section{Sosialisaatioresurssien teoria}
\label{luku-SRT-teoria}

\textcite{saks-gruman-2012} esittelevät sosialisaatioresurssien teorian (engl. \textit{Socialization Resources Theory,} SRT), joka on tapa jäsentää organisatorista sosialisaatiota ja perehtymistä. Se keskittyy nimenomaan niihin resursseihin, joita uudet työntekijät tarvitsevat sopeutuakseen onnistuneesti työyhteisöönsa ja -rooliinsa. Sen mukaan siirtymä uuteen työhön tai rooliin on aina haastava ja stressaava prosessi, josta selviytymiseen tulokkaat tarvitsevat näitä resursseja. Teoria koostuu seitsemästätoista ulottuvuudesta, jotka tukevat tulokkaita sopeutumisessa. Ulottuvuudet jaetaan neljään eri aikaulottuvuuteen. Ulottuvuudet on esitelty taulukossa \ref{tbl:srt-ulottuvuudet}. \parencite{saks-gruman-2012}.

\begin{table}[h]
    \footnotesize
    \begin{tabular}{llll}
        \toprule
        \textbf{Aikaulottuvuus} & \textbf{Nro} & \textbf{Ulottuvuus} \\
        \toprule
        Ennen työsuhteen alkua & 1. & Ennakoiva sosialisaatio \\
        \midrule
        Heti työsuhteen alettua & 2. & Muodollinen orientaatiojakso \\
        & 3. & Oma-aloitteisuuteen kannustaminen \\
        & 4. & Mentorointi \\
        \midrule
        Orientaation jälk. sosiaalisen pääoman resurssit & 5. & Sosiaaliset tapahtumat \\
        & 6. & Sosialisaatioagentit \\
        & 7. & Esihenkilön tuki \\
        & 8. & Sosiaaliset suhteet \\
        \midrule
        Orientaation jälk. työhön liittyvät resurssit & 9. & Työn tekemisen resurssit \\
        & 10. & Työn suunnittelu \\
        & 11. & Muodollinen työnopetus \\
        & 12. & Työtehtävät ja työn luonne \\
        & 13. & Informaatio \\
        & 14. & Palaute \\
        & 15. & Tunnustus ja arvostus \\
        \midrule
        Muodollisen perehdytyksen jälkeen & 16. & Seuranta \\
        & 17. & Perehdytysprosessin arviointi \\
        \bottomrule
    \end{tabular}
    \caption{Sosialisaatioresurssien teorian ulottuvuudet \parencite{saks-gruman-2012}}
    \label{tbl:srt-ulottuvuudet}
\end{table}

Sosialisaatioresurssien teorian mukaan toimenpiteet voidaan aloittaa jo ennen työsuhteen alkamista esimerkiksi esihenkilön puhelinsoitolla uudelle alaiselle. Näitä kutsutaan teoriassa ennakoivaksi sosialisaatioksi (engl. \textit{anticipatory socialization}). \parencite{saks-gruman-2012}.

Työsuhteen alkamisen jälkeen taas koittaa seuraava vaihe, johon kuuluvat muodollinen orientaatiojakso (engl. \textit{formal orientation}), oma-aloitteisuuteen kannustaminen (\textit{proactive encouragement}) ja mentorin nimeäminen (\textit{formal assistance}). \parencite{saks-gruman-2012}.

Sosialisaatioteorian mukaan orientaatiota seuraa noin kuuden kuukauden jakso, jonka aikana työntekijälle tarjotaan resursseja yhtäältä työyhteisössä toimimisen sosiaalisiin ulottuvuuksiin ja toisaalta työn tekemiseen liittyen. Näistä ensimmäiset eli sosiaalisen pääoman resurssit sisältävät sosiaalisia tapahtumia (engl. \textit{social events}), sosialisaatioagenttien apua (\textit{socialization agents}), esihenkilön tukea (\textit{supervisor support}) ja sosiaalisten suhteiden kehittämistä (\textit{relationship development}). Työhön liittyvät resurssit taas liittyvät työn tekemiseen (engl. \textit{job resources}), sen suunnitteluun (\textit{personal planning}), muodolliseen työnopetukseen (\textit{training}), työtehtäviin ja työn luonteeseen (\textit{assignments}), informaatioon (\textit{information}), palautteeseen (\textit{feedback}) sekä tunnustukseen ja arvostukseen (\textit{recognition and appreciation}). \parencite{saks-gruman-2012}.

Varsinaista perehdytysjaksoa taas seuraa jälkiseuranta, jossa työntekijän sopeutumista seurataan työnantajan aloitteesta (engl. \textit{follow-up}) sekä perehdytysprosessin arviointi (\textit{program evaluation}) \parencite{saks-gruman-2012}.



\section{Aiemmat systemaattiset kirjallisuuskatsaukset}

\textcite{garousi-ym-2020} tutkivat systemaattisessa kirjallisuuskatsauksessaan railoa ohjelmistokehitysalan yliopistokoulutuksen ja työelämän välillä. Näkökulmana oli koulutuksen kehittäminen vastaamaan työelämän tarpeita nykyistä paremmin. Tulosten perusteella merkittävimmät puutteet vasta-alkajien osaamisessa liittyivät testaamiseen, laatuun, projektinhallintaan, vaatimusmäärittelyyn sekä ohjelmistoalan ammattikäytäntöjen ja prosessien osaamiseen. Lisäksi \textcite{garousi-ym-2020} havaitsivat, että katsauksen mukaan teknisten taitojen lisäksi tarvitaan yhteistyöhön, kommunikaatioon, johtajuuteen ja kriittiseen ajatteluun liittyviä taitoja. 

\textcite{steinmacher-ym-2015} taas tutkivat vasta-alkajien kohtaamia haasteita avoimen lähdekoodin projekteissa. Katsauksen tuloksista todetaan, että haasteet voidaan jakaa viiteen kategoriaan: vuorovaikutus, tulokkaiden aiempi tietämys, alkuun pääseminen, dokumentaatio ja tekniset ongelmat.  Osallistumisen esteitä arvioitiin myös niiden syyn perusteella. Esteen syynä saattoi olla tulokas itse, kehitettävä tuote tai ympäröivä yhteisö. Tässä katsauksessa huomio oli siis nimenomaan avoimen lähdekoodin yhteisöissä.

Tässä katsauksessa pyritään siis täydentämään aiempien katsausten antamaa yleiskuvaa nimenomaan yritysten käyttämiin perehdyskäytäntöihin liittyen. Systemaattista kirjallisuuskatsausta tai -kartoitusta juuri tähän liittyen ei vaikuta aiemmin julkaistun ACM Digital Library-, IEEExplore tai Scopus-tietokannoissa. 

\chapter{Tutkimusprotokolla}

\textcite{kitchenham-charters-2007} toteavat, että systemaattisen kirjallisuuskatsauksen kolme vaihetta ovat suunnittelu, toteutus ja raportointi. Näistä suunnitteluvaihe sisältää tutkimusprotokollan (engl. \textit{review protocol}) laatimisen. Tutkimusprotokolla dokumentoi kirjallisuuskatsauksen toteuttamisen suunnitelman. Se tekee näkyväksi tutkimuksen toteuttamisen aikana tehtäviä valintoja sekä parantaa tutkimuksen toistettavuutta ja laatua. Opinnäytteen ollessa kyseessä tutkimusprotokolla myös hyväksytetään opinnäytteen ohjaajalla ja sitä tarkennetaan palautteen perusteella. \parencite{kitchenham-charters-2007}. Seuraavassa esitellään tämän tutkimuksen tutkimusprotokolla Kitchenhamin ja Chartersin esittämän rakenteen mukaan.

\section{Tutkimuskysymys}
\label{luku:tutkimuskysymys}

Tämän tutkimuksen tutkimuskysymys on: \textit{Minkälaisia käytäntöjä ohjelmistokehitystä tekevissä organisaatioissa käytetään ohjelmistokehittäjien perehdyttämiseksi työhönsä?}

\textcite{kitchenham-charters-2007} viittaavat Petticrew'hun ja Robertsiin, jotka ovat esitelleet ohjeet systemaattisten kirjallisuuskatsausten toteuttamiseen yhteiskuntatieteissä ja erityisesti PICOC-struktuurin (Population, Intervention, Comparison, Outcome, Context), jonka avulla voidaan muodostaa tutkimuskysymyksiä \parencite{petticrew-roberts-2006}. Näiden ohjeiden pohjalta \textcite{kitchenham-charters-2007} esittävät struktuuriin tarkennuksia nimenomaan ohjelmistokehityksen kirjallisuuskatsauksiin liittyen. 

\textcite{kitchenham-charters-2007} jäsentävät PICOC-struktuuria seuraavasti: \textit{Population} eli populaatio viittaa tutkittavaan ihmisryhmään, ammattinimikkeeseen, sovellusalaan tai ohjelmistokehitysalan organisaatioiden osajoukkoon (kuten pieniin yrityksiin tai ICT-alan yrityksiin). \textit{Intervention} eli interventio taas viittaa metodologiaan, teknologiaan, työkaluun tai käytänteeseen, joka vastaa johonkin tarpeeseen.\textit{Comparison}-ulottuvuudessa eli vertailussa taas on kyse siitä, mihin interventioon tutkimuksen kohteena olevaa interventiota verrataan. \textit{Outcomes} eli tulokset tarkoittavat ohjelmistokehityksen kontekstissa niitä oleellisia tuloksia, joiden saavuttamista tai säilyttämistä interventiolla tavoitellaan. \textit{Context} eli konteksti tarkentaa vielä tutkimuksen kohteen asiayhteyttä: keitä tutkimuksen osallistujat ovat? Mikä on tutkimuksen toimintaympäristö (esim. tiedeyhteisö vai yritykset)? \parencite{kitchenham-charters-2007}

Tämän tutkimuksen tutkimuskysymyksessä populaationa ovat tulokasohjelmistokehittäjät. Interventiona on työsuhteen alussa työnantajan aloitteesta tapahtuva työhön perehdyttäminen työpaikalla. Kontekstina on ohjelmistokehityksen ala. 

\section{Hakulauseke}

Tämän kirjallisuuskatsauksen aineiston hankinnassa käytetty hakulauseke muodostettiin luvussa  \ref{luku:tutkimuskysymys} esitellyn PICOC-struktuurin avulla. Hakulausekkeen osat on esitetty taulukossa \ref{tbl:picoc-ulottuvuudet}.

\begin{table}[h]
    \footnotesize
    \begin{tabular}{lp{0.66\textwidth}}
        \toprule
        {PICOC-ulottuvuus} & Hakulausekkeen osa \\
        \midrule
        Population & {\tt entry level OR novice OR junior OR newcomer OR new hire OR apprentice* OR "new team member", } \\
        & {\tt programmer OR developer OR engineer} \\
        \midrule
        Intervention & {\tt onboarding OR training OR mentoring} \\
        \midrule
        Comparison & - \\
        \midrule
        Outcome & - \\
        \midrule
        Context & {\tt software} \\
        \bottomrule
    \end{tabular}  
    \caption{PICOC-struktuurin ulottuvuudet ja niihin perustuvat hakulausekkeen osat}
    \label{tbl:picoc-ulottuvuudet}
\end{table}

Hakulausekkeen osia tarkennettiin useiden pilotointihakujen tuloksena. Esimerkiksi aluksi populaatiota kuvasi hakusanayhdistelmä \textit{entry level OR novice OR junior}, mutta sitä tarkennettiin pilotoinnin edetessä. Pilotoinnissa tehtiin siis tiedonhakuja eri tietokantoihin, silmältiin joitakin artikkeleita ja niiden lähdeluetteloita. Näistä löytyi useita lupaavalta vaikuttavia aineistoja, joiden perusteella hakulausekkeen osia tarkennettiin vähitellen. Kirjallisuudessa samasta ilmiöstä käytetään erilaisia käsitteitä, joita lisättiin hakulausekkeen osaan {\tt OR}-operaattorin avulla.

Tiedonhaun pilotoinnin yhteydessä kokeiltin myös hakuja, joissa ilmeni hakusana \textit{practices}, sillä tämän tutkimuksen tavoitteenahan on selvittää nimenomaan perehdyttämisen käytäntöjä. Tiedonhaut kuitenkin osoittivat, että tämä olisi rajannut hakutulosten määrää niin, että oleelliselta vaikuttavia aineistoja olisi jäänyt tulosten ulkopuolelle. Myös \textit{orientation}-hakusanaa kokeiltiin, mutta sillä tulosten määrä kasvoi satoihintuhansiin. 

\section{Tietokannat}

\textcite{brereton-ym-2007} mukaan ohjelmistokehitysaiheisissa kirjallisuuskatsauksissa relevantteja sähköisiä tietokantoja ovat mm. IEEExplore sekä ACM Digital Library. \textcite{kitchenham-charters-2007} taas toteavat, että kattavuuden varmistamiseksi olisi syytä harkita myös SCOPUS-tietokantaa.

Tässä kirjallisuuskatsauksessa käytettiin IEEExplore-, ACM Digital Library- ja SCOPUS-tietokantoja. Aluksi käytössä olivat vain IEEExplore ja SCOPUS, mutta pilotointihakujen yhteydessä havaittiin, että myös ACM Digital Library-tietokannasta näyttäisi löytyvän tutkimuskysymyksen kannalta relevanttia aineistoa, joten se päätettiin ottaa mukaan.

\begin{table}[h]
    \footnotesize
    \begin{tabular}{llp{0.8\textwidth}}
        \toprule
        {Tietokanta} & Tuloksia & Hakulauseke \\
        \midrule
        ACM & 166 & {\tt "query": { Title:(entry level OR novice OR junior OR newcomer OR new hire OR apprentice* OR "new team member") AND Title:(programmer OR developer OR engineer) AND Title:(onboarding OR training OR mentoring) AND Title:(software) } "filter": { ACM Content: DL }  } \\
        \midrule
        IEEE & 281 & {\tt ((entry level OR novice OR junior OR newcomer OR new hire OR apprentice* OR "new team member") AND (programmer OR  developer OR engineer)) AND (onboarding OR training OR mentoring) AND software } \\
        \midrule
        SCOPUS & 197 & {\tt TITLE-ABS-KEY ("entry level"  OR  novice  OR  junior  OR  newcomer OR "new hire"  OR  apprentice* OR "new team member"  AND  programmer  OR  developer  OR  engineer  AND  onboarding  OR  training  OR  mentoring  AND  software)  } \\
        \bottomrule
    \end{tabular}  
    \caption{Tietokannat, hakutulosten määrät ja käytettävät hakulausekkeet}
    \label{tbl:tietokannat}
\end{table}

ACM Digital Library- tai IEEExplore-tietokantojen hakutyökaluissa ei ollut mahdollista hakea vain aineistojen otsikoista, abstrakteista ja asiasanoista. IEEExplore-tietokannan kohdalla päätettiin kohdistaa haku kaikkiin kenttiin. ACM Digital Library-tietokannassa haku kohdistettiin vain aineistojen otsikoihin, sillä kaikista kentistä hakeminen olisi johtanut lähes kahteensataantuhanteen hakutulokseen. Tämän vuoksi myös ACM Digital Library-tietokannassa käytetty hakulauseke poikkeaa hieman muissa tietokannoissa käytetyistä hakulausekkeista. Tietokannat, tulosten määrä ja hakulausekkeet on esitelty taulukossa \ref{tbl:tietokannat}.

\section{Valintakriteerit}

Tässä kirjallisuuskatsauksessa lähdeaineistolle määriteltiin seuraavat hyväksymiskriteerit:

\begin{itemize}
    \item vastaa tutkimuskysymykseen eli käsittelee ohjelmistokehittäjien perehdytyskäytäntöjä
    \item englannin- tai suomenkielinen
    \item täysversio oltava saatavilla maksutta sähköisesti
    \item akateeminen artikkeli
    \item julkaisuvuosi aikaisintaan 2000
\end{itemize}

Hylkäämiskriteerit taas olivat:

\begin{itemize}
    \item käsittelee muita aiheita
    \item muu kuin englannin- tai suomenkielinen
    \item täysversio ei saatavilla sähköisesti Jyväskylän yliopiston lukuoikeuksilla
    \item artikkeli ei ole alkuperäinen tutkimus
    \item artikkeli ei ole primääritutkimus
    \item artikkeli esittelee opinnäytteen
    \item julkaistu ennen vuotta 2000
\end{itemize}

Aineisto päätettiin hyväksyä mukaan tutkimukseen vain silloin, jos kaikki hyväksymiskriteerit täyttyvät. Hylkäämiskriteereistä yhdenkin täyttymisen päätettiin johtavan aineiston hylkäämiseen.


%(Huomautus: \parencite{kitchenham-charters-2007} suosittelevat (luku 6.2.2), että sen jälkeen kun selkeästi epäolennaiset tutkimukset on jätetty pois, ja aletaan käydä tutkimuksia tarkemmin läpi, olisi hyvä pitää yllä listaa tutkimuksista, jotka on päätetty jättää katsauksen ulkopuolelle.

%\parencite{kitchenham-charters-2007} (luku 6.2.3) mukaan pois jätettyjä voi uudelleenarvioida ohjaajan  kanssa tai tekemällä test-retest arvion: "A single researcher (such as a PhD student) should consider discussing included and excluded papers with their advisor, an expert panel or other researchers. Alternatively, individual researchers can apply a test-retest approach, and re-evaluate a random sample of the primary studies found after initial screening to check the consistency of their inclusion/exclusion decisions."
%\textbf{TODO tehdäänkö ylläolevaa?}

\section{Valintamenettely}

Tiedonhaut tehtiin tietokantoihin 22.12.2022. Hakutulokset tallennettiin .bib-tiedostoiksi, jotka tuotiin Zotero-ohjelmaan, missä kaksoiskappaleet poistettiin. Jäljelle jäi 587 hakutulosta, jotka vietiin Notion-ohjelmistoalustalle perustettuun tietokantaan, jotta niiden käsittely olisi suoraviivaista ja systemaattista. Tietokantaan lisättiin luvussa \ref{luku-tiedonkeruustrategia} esitellyt tiedonkeruustrategian mukaiset kentät.

Valintakriteereitä sovellettiin tutustumalla aineistoon vähitellen syventyen. Aluksi  aineistosta rajattiin pois kirjat (5 kpl), joten jäljelle jäi 582. Seuraavaksi poistettiin vuotta 2000 vanhempi aineisto, mikä jälkeen jäljellä oli 527 artikkelia. Tämän jälkeen artikkelien nimet luettiin, minkä perusteella hylättiin 287 artikkelia. Nyt jäljelle jäi 240 artikkelia. Seuraavalla kierroksella silmäiltiin myös abstrakteja, minkä tuloksena 136 artikkelia hylättiin ja jäljelle jäi 104. Varovaisuusperiaatteen mukaisesti vain selkeästi jonkin hylkäämiskriteerin täyttävät artikkelit hylättiin.

Seuraavalla kierroksella luettiin abstrakteja ja artikkeleita. Hylkäämiskriteerejä sovellettiin edelleen, minkä tuloksena jäljelle jäi 32 artikkelia, joista haettiin täysversiot pdf-tiedostoina. Yhdestä täysversiota ei ollut saatavilla, joten se hylättiin. Hylkäämiskriteerien soveltamisen jälkeen jäljellä tässä vaiheessa oli siis 31 artikkelia. Nyt tarkastelun näkökulmaksi vaihdettiin hylkäämiskriteerien sijaan hyväksymiskriteerien näkökulma ja artikkeleihin tutustuttiin tarkasti. Mikäli kaikki hyväksymiskriteerit täyttyivät, artikkeli hyväksyttiin mukaan tutkimukseen. Näin tapahtui 16 artikkelin kohdalla.

Aineistoa täydennettiin tekemällä yksi snowballing-kierros, jossa tutkimukseen valittujen artikkelien lähdeluetteloiden perusteella etsittiin lisää artikkeleita. Tämän tuloksena aineistoon lisättiin neljä artikkelia. Yhteensä katsaukseen valittiin siis kaksikymmentä artikkelia. Aineiston valintamenettely on havainnollistettu kuviossa \ref{kuvio:aineiston-rajaaminen-laaja}.

\begin{figure}[h]
    \centering
    \includegraphics[width=11cm]{media/aineiston-rajaaminen-laaja-vaakalinja.png}
    \caption{Artikkelien valintamenettely}
    \label{kuvio:aineiston-rajaaminen-laaja}
\end{figure}

Yleisin artikkelin hylkäämiseen johtanut syy oli "Muu aihe", joka kohti 422 artikkelin hylkäämiseen. Monissa artikkeleissa kontekstina oli avoimen lähdekoodin projektien perehdytyskäytännöt. Neljässäkymmenessäviidessä artikkelissa taas käsiteltiin ohjelmistokehittäjien osaamista siltä kannalta, miten oppilaitokset voisivat parantaa opetussuunnitelmiaan vastaamaan yritysten tarpeita. Näissä näkökulmana oli siis oppilaitoksissa tehtävä työ, ei yritysten perehdytyskäytännöt. Useat artikkelit taas liittyivät yleisesti ohjelmistokehittäjien osaamisen parantamiseen, mutta ei nimenomaan perehdyttämisprosessin aikana. Monissa artikkeleissa oli kehitetty oivaltavia uusia työkaluja tai lähestymistapoja siihen, miten perehdyttämistä voisi tukea tai parantaa, mutta tässä tutkimuksessa tavoitteena on tutkia nimenomaan olemassaolevia käytäntöjä yrityksissä.

\section{Tiedonkeruustrategia}
\label{luku-tiedonkeruustrategia}

\textcite{kitchenham-charters-2007} mainitsevat, että tiedonkeruustrategia (engl. \textit{data extraction strategy}) määrittää sen, miten kustakin primääritutkimuksesta vaadittavat tiedot saadaan. Tämän tutkimuksen tiedonkeruustrategiaa tarkennettiin vaiheittain. Ensimmäinen versio strategiasta oli suuntaa-antava. Sitä pilotoitiin keräämällä dataa kolmesta potentiaalisesti katsaukseen soveltuvasta artikkelista. Pilotoinnin perusteella tiedonkeruustrategiaa täydennettiin merkittävästi lisäämällä siihen useita lisäkenttiä kuten tiedot tutkimuksen kohderyhmistä, kontekstista ja tuloksista.

Seuraavaksi esitellään tiedonkeruustrategia kokonaisuudessaan. Aluksi jokaisesta artikkelista kerättiin lähdetietokannan tarjoama BibTex-tietue, joka sisälsi mm. seuraavat tiedot:

\begin{itemize}
    \item artikkelin nimi
    \item kirjoittajat
    \item DOI-osoite
    \item julkaisuvuosi
    \item abstrakti
\end{itemize}

Tietuetta täydennettiin seuraavilla tiedoilla:

\begin{itemize}
    \item hylkäyskriteeri
    \item jatkotutkimusaiheet
    \item tutkimuksen kohderyhmä
    \item tutkimuksen konteksti
    \item kohdehenkilöiden lukumäärä (mikäli sovellettavissa)
    \item artikkelissa mainitut perehdytyskäytännöt
    \item tiedonkeruun status (tiedot kerätty / kesken / ei kerätty)
    \item tutkimuksen tulokset
    \item tutkimusmenetelmät
    \item tutkimustyyppi
    \item "twiitti" eli lyhyt kuvaus tutkimuksesta
    \item valittu katsaukseen (kyllä / ei / ehkä)
\end{itemize}

\section{Tiedonkeruumenettely}

Artikkelien valitsemisen jälkeen suoritettiin tiedonkeruuvaihe, jossa artikkelit luettiin ja niistä kerättiin luvun \ref{luku-tiedonkeruustrategia} tiedonkeruustrategian mukaiset tiedot. Työkaluna käytettiin Notion-ohjelmistoalustaa, johon tiedot kerättiin sähköisessä muodossa. 

Tiedonkeruuvaiheessa pyrittiin tekemään mahdollisimman vähän tulkintaa, joten kerättiin vain sellainen tieto, jonka artikkelin kirjoittajat olivat implisiittisesti artikkelissa maininneet. Tavoitteena oli säilyttää data mahdollisimman autenttisena katsauksen laadun varmistamiseksi. Vasta kun tiedot oli kerätty kaikista artikkeleista, arvioitiin kerättyjä tietoja kokonaisuutena. Tämä johti joidenkin tietojen yhtenäistämiseen. Esimerkiksi eri artikkeleissa oli mainittu mentorointiin liittyviä käytäntöjä termeillä \textit{"mentor"}, \textit{"buddy"} ja \textit{"tutor"}. Datassa nämä yhtenäistettiin käytännön \textit{"mentorointi"} alle. Eri artikkeleissa oli mainittu myös tulokkaan perehtyminen yrityksen sisäiseen intranettiin, wikiin tai muuhun vastaavaan tietovarastoon. Nämä käytännöt yhtenäistettiin käsitteellä \textit{“sisäiseen dokumentaatiooon perehtyminen”}.

Myös muita kerättyjä tietoja yhtenäistettiin ja selkeytettiin - esimerkiksi \textit{"juniorit"} saattoi olla merkitty sekä tutkimuksen kontekstiksi että kohderyhmäksi tai \textit{"katselmointi"} sekä tutkimuksen kontekstiksi että perehdytyskäytännöksi. Näissä tilanteissa tiedot yhdenmukaistettiin.

Käsitteellisesti toisiaan muistuttavien tietojen eroja tulkittiin joko yhdistämällä kaksi käsitettä (kyselytutkimusten osalta tutkimusmenetelmäksi merkitty \textit{"avoimet kysymykset"} yleistettiin \textit{kysely}-menetelmän alle) tai tulkitsemalla käsitteiden eroja: kontekstien \textit{"etätyöskentely"} ja \textit{"globaalisti hajautettu ohjelmistokehitys"} eroiksi tulkittiin se, että jälkimmäisessä hajauttaminen on tehty tarkoituksella ja etätyöskentelyssä olosuhteiden pakosta. Etätyöskentely toki voi olla vapaaehtoistakin, mutta tämän katsauksen artikkeleissa sitä tehtiin pandemiaan \textcite{rodeghero-ym-2021} tai työlupiin \textcite{hemphill-begel-2011} liittyvien haasteiden vuoksi. \textcite{britto-ym-2020} tutkivat Ericsson AB:lle palkattujen intialaisten ja \textcite{moe-ym-2020} norjalaiseen pankkiin palkattujen portugalilaisten ohjelmistokehittäjien käytäntöjä - näissä kontekstiksi merkittiin \textit{"globaalisti hajautettu ohjelmistokehitys"}.

\chapter{Katsauksen tulokset}

TODO metateksti

\section{Katsaukseen valitut artikkelit}

Tähän systemaattiseen kirjallisuuskatsaukseen valittiin siis yhteensä 20 artikkelia. Artikkelit on esitelty taulukossa \ref{tbl:artikkelit}. Artikkeleista kuusi oli ilmestynyt tietojenkäsittelytieteen alan journal-lehdissä. Kolmetoista artikkelia taas oli julkaistu tieteellisten konferenssien kokoomateoksissa (engl. \textit{proceedings}). Yksi artikkeli \parencite{hemphill-begel-2011} on ilmestynyt Microsoft Technical Report -julkaisusarjassa. 

Artikkelit on julkaistu vuosina 2005-2021. Valtaosa artikkeleista, viisitoista kappaletta, on julkaistu vuoden 2015 jälkeen. Tarkat kappalemäärät julkaisuvuosittain on esitelty kuviossa \ref{kuvio:kappalemaarat-julkaisuvuosittain}.

\begin{figure}[h]
    \centering
    \includegraphics[width=\textwidth]{media/kappalemaarat-julkaisuvuosittain.png}
    \caption{Artikkeleiden kappalemäärät julkaisuvuosittain}
    \label{kuvio:kappalemaarat-julkaisuvuosittain}
\end{figure}




\begin{table}\fontsize{10}{7.2}\selectfont
\begin{tabular}{p{3.5cm}p{12cm}}
        & Artikkelin nimi  \\
\toprule

 \textcite{rodeghero-ym-2021} & Please Turn Your Cameras On: Remote Onboarding of Software Developers during a Pandemic \\

\midrule
\textcite{azanza-ym-2021} & Onboarding in Software Product Lines: Concept Maps as Welcome Guides \\

\midrule
\textcite{ju-ym-2021} & A Case Study of Onboarding in Software Teams: Tasks and Strategies \\

\midrule
\textcite{britto-ym-2020} & Evaluating and strategizing the onboarding of software developers in large-scale globally distributed projects\\

\midrule
\textcite{yates-ym-2020} & Characterizing the transfer of program comprehension in onboarding: an information-push perspective\\

\midrule
\textcite{moe-ym-2020} & Studying Onboarding in Distributed Software Teams: A Case Study and Guidelines \\

\midrule
\textcite{kumar-wallace-2019} & Patterns of Identity and Interaction in an Agile Community of Practice \\

\midrule
\textcite{viviani-murphy-2019} & Reflections on Onboarding Practices in Mid-Sized Companies \\

\midrule
\textcite{buchan-ym-2019} & Effective team onboarding in Agile software development: techniques and goals \\

\midrule
\textcite{tuzun-ym-2018} & Are computer science and engineering graduates ready for the software industry?: experiences from an industrial student training program \\

\midrule
\textcite{matturro-ym-2017} & Difficulties of Newcomers Joining Software Projects Already in Execution \\

\midrule
\textcite{britto-ym-2017} & Onboarding software developers and teams in three globally distributed legacy projects: A multi-case study \\

\midrule
\textcite{pham-ym-2017} & Onboarding inexperienced developers: struggles and perceptions regarding automated testing \\

\midrule
\textcite{kumar-ym-2016} & Mentoring trajectories in an evolving agile workplace \\

\midrule
\textcite{shannon-pool-2016} & Agile Processes, in Software Engineering, and Extreme Programming \\

\midrule
\textcite{viana-ym-2014} & Knowledge transfer between senior and novice software engineers: A qualitative analysis \\

\midrule
\textcite{hemphill-begel-2011} & Not Seen and Not Heard: Onboarding Challenges in Newly Virtual Teams \\

\midrule
\textcite{kulkarni-ym-2010} & From Student to Software Engineer in the Indian IT Industry: A Survey of Training \\

\midrule
 \textcite{johnson-senges-2010} & Learning to be a programmer in a complex organization: A case study on practice-based learning during the onboarding process at Google\\

\midrule
\textcite{bjornson-dingsøyr-2005} & A Study of a Mentoring Program for Knowledge Transfer in a Small Software Consultancy Company \\

\bottomrule
\label{tbl:artikkelit}
\end{tabular}
\caption{Artikkelit}
\end{table}

\section{Perehdytyskäytännöt}

Artikkeleissa mainittiin yhteensä 45 erilaista perehdytyskäytäntöä. Mentorointi mainittiin kuudessatoista artikkelissa tulokkaiden perehdytyskäytäntönä. Koodin katselmointi (eng. \textit{code review}) mainittiin yhdeksässä artikkelissa, kuten myös yhteistoiminnallinen ohjelmointi (pari- tai ryhmäohjelmointi). Organisaation sisäiseen dokumentaatioon perehtyminen oli kirjattu käytännöksi seitsemään eri artikkeliin. Tulokkaiden toiminta vertaisryhmänä, työtehtävän kontekstualisointi ja erilaiset tarkistuslistat saivat kukin viisi mainintaa, kuten myös "Good First Issue", joka viittaa työtehtävään, joka on ennakolta määritelty tulokkaille erityisen hyvin sopivaksi. Kuviossa \ref{kuvio:kaytannot} esitellään kaikki käytännöt esiintymismäärineen.

\begin{figure}[h]
    \centering
    \includegraphics[width=14cm]{media/45-kaytannot.png}
    \caption{Perehdyttämiskäytäntöjen esiintymismäärät}
    \label{kuvio:kaytannot}
\end{figure}

Artikkeleissa mainittuja perehdyttämiskäytäntöjä jaoteltiin luvussa \ref{luku-SRT-teoria} esitellyn sosialisaatioresurssien teorian mukaisesti. Jaottelu on havainnollistettu taulukossa \ref{tbl:srt-ulottuvuudet-ja-niiden-kaytannot}. Esimerkiksi sosialisaatioresurssien teorian ulottuvuuteen \textit{"1. Ennakoiva sosialisaatio"} sijoitettiin alkukartoituskyselyihin ja esittäytymissähköposteihin liittyvät käytännöt. Ulottuvuuksien mainintamäärät on kuvattu kuviossa \ref{kuvio:ulottuvuudet}. Mentoroinnin ulottuvuus oli yleisin, sillä se mainittiin kuudessatoista artikkelissa. Ulottuvuus 12 (työtehtävät ja työn luonne) mainittiin viidessätoista. Nämä perehdytyskäytännöt liittyivät läheisesti tulokkaan työtehtävien tekemiseen käytännössä - esimerkiksi tulokkaalle etukäteen valmistellut projektit, tulokkaan laatimat viikkoraportit ja teknisiin kokouksiin osallistuminen. Seuraavaksi eniten mainintoja sai palautteen ulottuvuus (11 artikkelia), johon luettiin katselmointi, tulokkaalle annettava positiivinen palaute ja tulokkaan suoriutumisen arviointi.

% \tiny	\scriptsize \footnotesize \small

\begin{table}[h]
    \tiny
    \begin{tabular}{llll}
        \toprule
        \textbf{Nro} & \textbf{Ulottuvuus} & \textbf{Perehdytyskäytäntö} \\
        \toprule
        1. & Ennakoiva sosialisaatio & Alkukartoituskysely \\
        & & Esittäytymissähköposti \\
        \midrule
        2. & Muodollinen orientaatiojakso & Muodollinen orientaatiojakso\\
        \midrule
        3. & Oma-aloitteisuuteen kannustaminen & Aktiivisuuteen kannustaminen \\
        & & Tulokas haastattelee kokeneita \\
        \midrule
        4. & Mentorointi & Mentorointi \\
        \midrule
        5. & Sosiaaliset tapahtumat & Esittäytymiskokous \\
        & & Sosiaaliset tapahtumat\\
        \midrule
        6. & Sosialisaatioagentit & Avun tarjoaminen etukäteen \\
        \midrule
        7. & Esihenkilön tuki & Esihenkilön tuki \\
        & & 1:1-tapaamiset \\
        \midrule
        9. & Työn tekemisen resurssit & Sisäiseen dokumentaatioon perehtyminen \\
        & & Ohjelmointikäytännöt määritelty \\
        \midrule
        10. & Työn suunnittelu & Tarkastuslistat\\
        & & Odotusten sanallistaminen \\
        & & Tavoitteiden asettaminen ja arviointi\\
        \midrule
        11. & Muodollinen työnopetus & Kurssit \\
        & & Muodollinen koulutusjakso \\
        & & Työyhteisötaitokoulutus\\
        & & Codelabs-tutoriaali\\
        \midrule
        12. & Työtehtävät ja työn luonne & Työtehtävän kontekstualisointi \\
        & & Good First Issue \\
        & & Etukäteen valmisteltu projekti \\
        & & Viikkoraportti \\
        & & Koodin tuottaminen itsenäisesti tuettuna\\
        & & Teknisiin kokouksiin osallistuminen\\
        & & Retrospektiivit\\
        & & Yhteistoiminnallinen ohjelmointi\\
        \midrule
        13. & Informaatio & Työnkuvauksen esittely\\
        & & Arkkitehtuurin esittely\\
        & & Käsitekartat\\
        & & Olemassaolevan koodin esittely\\
        \midrule
        14 & Palaute & Katselmointi \\
        & & Positiivinen palaute tulokkaalle\\
        & & Tulokkaan suoriutumisen arviointi\\
        \midrule
        17 & Perehdytysprosessin arviointi & Palaute tulokkaalta\\
        \midrule
        18 & Orientaation jälkeinen tulokkaiden keskinäinen oppiminen & Tulokkaat toimivat vertaisryhmänä\\
        & & Yhteistyötä vaativien työtehtävien antaminen \\
        & & Yhteinen perehdytysprosessi useille tulokkaille\\
        & & Lukupiiri\\
        \midrule
        & Ei luokiteltavissa & Sisäinen CV \\
        & & Tulokkaat havainnoivat kokeneita\\
        & & Seniorit mukana rekrytoinnissa\\
        & & Yksittäinen perehdytys yhdelle tulokkaalle \\
        & & Itsenäinen oppiminen\\
        \bottomrule
    \end{tabular}
    \caption{Artikkeleissa havaitut käytännöt sosialisaatioresurssiteoriaan \parencite{saks-gruman-2012} pohjautuen jaoteltuna}
    \label{tbl:srt-ulottuvuudet-ja-niiden-kaytannot}
\end{table}




\begin{figure}[h]
    \centering
    \includegraphics[width=\textwidth]{media/ulottuvuudet.png}
    \caption{Perehdyttämiskäytäntöjen esiintymismäärät jaoteltuna sosialisaatioresurssien \parencite{saks-gruman-2012} mukaan}
    \label{kuvio:ulottuvuudet}
\end{figure}

Neljä perehdytyskäytäntöä taas liittyi selvästi orientaatiovaiheen jälkeiseen tulokkaisen keskinäiseen yhteistoiminnalliseen oppimiseen. Näissä käytännöissä samaan aikaan organisaatioon liittyneet tulokkaat toimivat siis yhdessä. Heille saatettiin antaa yhteistyötä vaativia työtehtäviä. Eräässä organisaatiossa yksi perehdytyskäytännöistä oli lukupiiri, jossa käsiteltiin ohjelmistokehitykseen liittyviä artikkeleita. Nämä yhteistoiminnallista oppimista edistävät käytännöt olen sijoittanut ulottuvuuteen 18, joka täydentää sosialisaatioresurssien teorian seitsemäätoista ulottuvuutta. Kuten kuviosta \ref{kuvio:ulottuvuudet} nähdään, tämä ulottuvuus on saanut neljänneksi eniten mainintoja kaikista ulottuvuuksista. Siihen kuuluvia käytäntöjä on mainittu yhdeksässä artikkelissa. 

Viisi perehdytyskäytäntöä eivät olleet luokiteltavissa tiettyyn sosialisaatioresurssien teorian ulottuvuuteen. Nämä olivat mm. sisäinen CV (jolla tulokkaat tekivät omaa osaamistaan näkyväksi, mutta myös saivat tietoa kollegoidensa osaamisesta), tulokkaiden tekemä kokeneiden työntekijöiden havainnointi sekä kokeneiden työntekijöiden mukanaolo rekrytointiprosessissa. 

Tämän katsauksen artikkeleista ei löytynyt yhtään perehdyttämiskäytäntöä, jotka liittyisivät tunnustukseen ja arvostukseen (ulottuvuus 15) tai seurantaan (ulottuvuus 16). Myöskään ulottuvuuteen 8 (sosiaaliset suhteet) liittyviä käytäntöjä ei artikkeleista havaittu.

Kaikki artikkelit ja niissä havaittujen perehdytyskäytäntöjen ulottuvuudet on esitelty matriisissa \ref{tbl:ulottuvuusmatriisi}. TODO \ref{tbl:longtable-ulottuvuusmatriisi}

 \begin{longtable}[c]{| c | c | c | c | c | c | c | c | c | c | c | c | c | c | c | c | c |}
 
 \caption{Long table caption.\label{tbl:longtable-ulottuvuusmatriisi}}\\

 \hline
 \multicolumn{17}{| c |}{Begin of Table}\\
 \hline
 & \textbf{1} & \textbf{2} & \textbf{3} & \textbf{4} & \textbf{5} & \textbf{6} & \textbf{7} & \textbf{9} & \textbf{10} & \textbf{11} & \textbf{12} & \textbf{13} & \textbf{14} & \textbf{17} & \textbf{18} & \textbf{19} \\
 \hline
 \endfirsthead

 \hline
 \multicolumn{17}{|c|}{Continuation of Table \ref{long}}\\
 \hline
& \textbf{1} & \textbf{2} & \textbf{3} & \textbf{4} & \textbf{5} & \textbf{6} & \textbf{7} & \textbf{9} & \textbf{10} & \textbf{11} & \textbf{12} & \textbf{13} & \textbf{14} & \textbf{17} & \textbf{18} & \textbf{19} \\
 \hline
 \endhead

 \hline
 \endfoot

 \hline
 \multicolumn{17}{| c |}{End of Table}\\
 \hline\hline
 \endlastfoot

\textcite{rodeghero-ym-2021} & x &  &  & x & x & x & x &  &  &  & x &  &  &  & x &  \\
\textcite{azanza-ym-2021}    &  &  &  &  &  &  &  &  &  &  &  & x &  &  &  &  \\
\textcite{ju-ym-2021}        &  &  &  & x &  &  &  &  &  &  & x &  & x &  & x &  \\
\textcite{britto-ym-2020}    &  &  &  & x &  &  &  & x & x &  &  &  & x &  & x & x \\
\textcite{yates-ym-2020}     &  &  &  &  &  &  &  &  &  &  &  & x &  &  &  &  \\
\textcite{moe-ym-2020}       &  & x &  & x & x &  & x &  &  &  & x &  & x & x & x &  \\
\textcite{kumar-wallace-2019} &  &  &  & x &  &  &  & x &  &  & x & x & x &  &  & x \\
\textcite{viviani-murphy-2019} &  &  &  & x &  &  &  &  &  &  & x & x & x &  &  &  \\
\textcite{buchan-ym-2019} &  & x &  & x & x &  & x & x & x & x & x & x &  &  &  & x \\
\textcite{tuzun-ym-2018} & x &  &  &  &  &  &  &  &  & x & x &  &  & x &  &  \\
\textcite{matturro-ym-2017} &  &  & x & x &  &  & x & x & x &  & x &  &  &  &  &  \\
\textcite{britto-ym-2017} &  &  &  & x &  &  &  &  & x & x & x & x & x &  & x & x \\
\textcite{pham-ym-2017} &  &  &  & x &  &  &  &  &  &  & x &  & x &  & x &  \\
\textcite{kumar-ym-2016} &  &  &  & x &  &  &  & x &  &  & x & x & x &  & x & x \\
\textcite{shannon-pool-2016} &  &  & x & x & x &  &  &  &  & x & x &  & x &  & x &  \\
\textcite{viana-ym-2014} &  &  &  &  &  &  &  & x &  & x & x &  &  &  &  & x \\
\textcite{hemphill-begel-2011} &  & x & x & x &  &  &  &  & x &  & x &  &  &  &  &  \\
\textcite{kulkarni-ym-2010} &  &  &  & x &  &  &  &  &  & x &  &  & x &  & x & x \\
\textcite{johnson-senges-2010} &  & x & x & x &  &  &  & x & x & x & x &  & x &  &  & x \\
\textcite{bjornson-dingsøyr-2005} &  &  &  & x &  &  &  &  &  &  &  &  &  &  &  &  \\
yht & 2 & 4 & 4 & 16 & 4 & 1 & 4 & 7 & 6 & 7 & 15 & 7 & 11 & 2 & 9 & 8 \\
 
 \end{longtable}



\begin{table}[h]
    \scriptsize
    \begin{tabular}{lllllllllllllllll}

\toprule
& \textbf{1} & \textbf{2} & \textbf{3} & \textbf{4} & \textbf{5} & \textbf{6} & \textbf{7} & \textbf{9} & \textbf{10} & \textbf{11} & \textbf{12} & \textbf{13} & \textbf{14} & \textbf{17} & \textbf{18} & \textbf{19} \\

\midrule
\textcite{rodeghero-ym-2021} & x &  &  & x & x & x & x &  &  &  & x &  &  &  & x &  \\
\midrule
\textcite{azanza-ym-2021}    &  &  &  &  &  &  &  &  &  &  &  & x &  &  &  &  \\
\midrule
\textcite{ju-ym-2021}        &  &  &  & x &  &  &  &  &  &  & x &  & x &  & x &  \\
\midrule
\textcite{britto-ym-2020}    &  &  &  & x &  &  &  & x & x &  &  &  & x &  & x & x \\
\midrule
\textcite{yates-ym-2020}     &  &  &  &  &  &  &  &  &  &  &  & x &  &  &  &  \\
\midrule
\textcite{moe-ym-2020}       &  & x &  & x & x &  & x &  &  &  & x &  & x & x & x &  \\
\midrule
\textcite{kumar-wallace-2019} &  &  &  & x &  &  &  & x &  &  & x & x & x &  &  & x \\
\midrule
\textcite{viviani-murphy-2019} &  &  &  & x &  &  &  &  &  &  & x & x & x &  &  &  \\
\midrule
\textcite{buchan-ym-2019} &  & x &  & x & x &  & x & x & x & x & x & x &  &  &  & x \\
\midrule
\textcite{tuzun-ym-2018} & x &  &  &  &  &  &  &  &  & x & x &  &  & x &  &  \\
\midrule
\textcite{matturro-ym-2017} &  &  & x & x &  &  & x & x & x &  & x &  &  &  &  &  \\
\midrule
\textcite{britto-ym-2017} &  &  &  & x &  &  &  &  & x & x & x & x & x &  & x & x \\
\midrule
\textcite{pham-ym-2017} &  &  &  & x &  &  &  &  &  &  & x &  & x &  & x &  \\
\midrule
\textcite{kumar-ym-2016} &  &  &  & x &  &  &  & x &  &  & x & x & x &  & x & x \\
\midrule
\textcite{shannon-pool-2016} &  &  & x & x & x &  &  &  &  & x & x &  & x &  & x &  \\
\midrule
\textcite{viana-ym-2014} &  &  &  &  &  &  &  & x &  & x & x &  &  &  &  & x \\
\midrule
\textcite{hemphill-begel-2011} &  & x & x & x &  &  &  &  & x &  & x &  &  &  &  &  \\
\midrule
\textcite{kulkarni-ym-2010} &  &  &  & x &  &  &  &  &  & x &  &  & x &  & x & x \\
\midrule
\textcite{johnson-senges-2010} &  & x & x & x &  &  &  & x & x & x & x &  & x &  &  & x \\
\midrule
\textcite{bjornson-dingsøyr-2005} &  &  &  & x &  &  &  &  &  &  &  &  &  &  &  &  \\
\midrule
\midrule
yht & 2 & 4 & 4 & 16 & 4 & 1 & 4 & 7 & 6 & 7 & 15 & 7 & 11 & 2 & 9 & 8 \\
\bottomrule

\end{tabular}
\caption{Artikkeleissa havaittujen perehdytyskäytäntöjen sosialisaatioresurssiulottuvuudet TODO täytä}
\label{tbl:ulottuvuusmatriisi}
\end{table}

\subsection{Johtopäätökset}

Katsauksen tulosten perusteella voidaan siis todeta, että mentorointi vaikuttaa olevan tärkeässä asemassa ohjelmistokehittäjien perehdyttämisessä. Kuten \textcite{viviani-murphy-2019} toteavat, vahvasti mentorointiin nojaava perehdytyskäytäntö saattaa muodostua mentorille taakaksi. Myös \textcite{moe-ym-2020} toteavat, että mentoreiden saattaa olla haastavaa jakaa työaikaansa omien työtehtäviensä ja mentoroinnin välillä. \textcite{britto-ym-2017} täydentävät, että erityisesti pitkään käynnissä olleissa legacy-projekteissa tulokkaat joutuvat turvautumaan mentorin apuun usein, mikä vaikuttaa mentorien tuottavuuteen ja työtyytyväisyyteen. Tässä katsauksessa tunnistettu tulokkaiden keskinäisen yhteistoiminnallisen oppimisen ulottuvuus voisi keventää mentorien työtaakkaa: mikäli organisaatiossa aloittaa samaan aikaan useita uusia työntekijöitä, voisi voisi perehdyttämisprosessissa painottaa tulokkaiden keskinäistä yhteistoimintaa vertaisryhmänä. Yhteistyötä vaativat työtehtävät voisivat auttaa tulokkaita perehtymään organisaatiossa tehtävään työhön, kehitettäviin tuotteisiin ja olemassaolevaan ohjelmistokoodiin. Tulokkailla on myös paljon osaamista, jonka jakautumista vertaisryhmässä työskentely voisi edistää. Työskentelyssä ilmenevät pulmatilanteet tulokkaat voisivat pyrkiä ratkaisemaan ensin yhdessä ja konsultoimaan nimettyä mentoria vasta tarvittaessa. Tällaiset menetelmät toki edellyttävät, että sosialisaatioresurssien teorian yhdeksänteen ulottuvuuteen, työn tekemisen resursseihin, liittyvä valmistelu on tehty organisaatiossa huolellisesti. Esimerkiksi sisäisen teknisen dokumentaation tulisi olla ajantasaista ja riittävän tarkkaa.

Työtehtäviin ja työn luonteeseen liittyvät käytännöt ovat katsauksen mukaan myös oleellisia perehdyttämsiessä. Työyhteisössä voidaan varautua uuden työntekijän aloitukseen esimerkiksi valmistelemalla hänelle etukäteen projekti tai valitsemalla tulokkaalle erityisen hyvin sopivia, suoraviivaisia työtehtäviä (nk. \textit{Good First Issue}). Kuten \textcite{ju-ym-2021} toteavat, työtehtävien haastavuuden tulisi kuitenkin pian nousta, jotta tulokas saa haastavia tehtäviä, mikä puolestaan pitää yllä motivaatiota.

Kuten \textcite{britto-ym-2020} toteavat, että perehdytysprosessit tulee suunnitella huolellisesti. Mielenkiintoinen jatkotutkimusaihe olisi se, miten perehdytysprosesseja strukturoimalla ja suunnittelemalla voitaisiin edistää organisatorisen sosialisaation toivottuja tuloksia kuten työntekijöiden tyytyväisyyttä, työhön sitoutumista ja työssä suoriutumista.

Erityisen mielenkiintoista on tarkastella niitä sosialisaatioresurssien teorian ulottuvuuksia, joihin liittyviä käytäntöjä katsauksessa ei löydetty yhtään. Näitähän olivat tulokkaan tunnustukseen tai arvostukseen (\#15), perehdytysprosessin seurantaan (\#16) ja sosiaalisiin suhteisiin (\#8) liittyvät ulottuvuudet. 

Sosiaalisiin suhteisiin liittyvien käytäntöjen puutetta katsauksessa voidaan selittää sillä, että sosialisaatioresurssien teoriassa sitä sivuaa kaksi muuta ulottuvuutta: sosiaaliset tapahtumat (\#5) ja sosialisaatioagentit (\#6), joihin liittyviä käytäntöjä kyllä havaittiin. Sosiaalisiin suhteisiin liittyvät käytännöt viittaavat \textcite{wanberg-2012} mukaan erilaisia toimia, jotka auttavat tulokkaita muodostamaan sosiaalisia suhteita: esimerkiksi esittäytymiset ja lyhyet keskustelut luetaan tähän ulottuvuuteen. Näitä käytäntöjä ei siis katsauksessa havaittu. Tätä saattaa selittää se, että esittäytyminen uudelle kollegalle ja tutustumiseen tähtäävät keskustelunavaukset kuuluvat pikemminkin yleisiin sosiaalisiin työyhteisötaitoihin kuin varsinaisiin perehdyttämiskäytäntöihin. 

Perehdytysprosessin seurantaan liittyviä käytäntöjä ei myöskään siis katsauksessa havaittu. \textcite{wanberg-2012} toteavat seurannalla tarkoitettavan sitä, missä määrin organisaatio seuraa uusia työntekijöitä muodollisen perehdyttämisjakson päätyttyä kuten tarkistamalla työn sujumista tai tiedustelemalla avun tarpeita. Tämä onkin mielenkiintoinen piirre katsauksen aineistossa. Kun esihenkilön tukikin (ulottuvuus 7) mainittiin vain neljässä artikkelissa, olisi perusteltua selvittää, miten uusien työntekijöiden työskentely sujuu perehdyttämisen jälkeen, mikä voidaan mainita jatkotutkimusaiheena.

Tulokkaan arvostamisen ja tunnustuksen antamisen ulottuvuuteen liittyviä käytäntöjä ei myöskään löydetty lainkaan. \textcite{wanberg-2012} määrittelevät tämän ulottuvuuden liittyvän siihen, missä määrin uudet tulokkaat saavat tunnustusta ja kiitosta ponnisteluistaan ja suorituksistaan, mikä on heidän mukaansa erityisen tärkeää työsuhteen alkuvaiheessa, jolloin työntekijä pyrkii jäsentämään uutta ympäristöään ja rooliaan siinä. Ulottuvuuteen liittyy läheisesti myös palautteen ulottuvuus (\#14), johon liittyviä käytäntöjä mainittiin yhdessätoista artikkelissa. Nämä ulottuvuudet ovat käsitteellisesti lähellä toisiaan, joten on mahdollista, että tulokkaan arvostamiseen ja tunnustukseen liittyviä käytäntöjä on katsauksessa tulkittu palautteena. Aihetta olisi tarpeen tutkia tarkemmin paremman käsityksen saamiseksi. Ovathan palkitsemisjärjestelmät keskeisiä mekanismeja, jotka auttavat tulokkaiden sopeutumista organisaatioon \parencite{jokisaari-nurmi-2009}. Mahdollisia tutkimusaiheita voisivat olla se, miten tulokkaiden saama arvostus ja palaute vaikuttavat perehtymisprosessin tuloksiin tai se, minkälaisia käytäntöjä ohjelmistokehityksen arkeen voidaan tuoda, jotta positiivinen ja palaute koituisi kaikkien työryhmän jäsenten osaksi. Voisiko ketterien menetelmien kuten Scrumin säännöllisiä kokouksia rikastaa ääneenlausutulla arvostuksella ja kiitoksella?

Vain kahdessa artikkelissa mainittiin perehdytyskäytäntönä se, että tulokkaalle asetettaisiin tavoitteita, joiden saavuttamista arvioitaisiin. Mikä mahtaa selittää näin vähäistä määrää? Onko tavoitteiden asettaminen niin ilmiselvä käytäntö, ettei sitä artikkeleissa edes mainita, vai eikö tulokkaille todella aseteta tavoitteita? Lisätutkimusta tarvitaan asian selvittämiseksi.

\section{Tutkimustyypit ja -menetelmät}
\label{luku-tutkimustyypit-ja-menetelmat}

Katsaukseen valituissa artikkeleissa raportoitiin erilaista tutkimuksista (ks. \ref{kuvio:tutkimustyypit}). Kolmessatoista artikkelissa oli hyödynnetty yhtä tutkimustyyppiä ja seitsemässä kahta tai kolmea. Esimerkiksi \textcite{azanza-ym-2021} tutkivat kyselytutkimuksen avulla käsitekarttojen käyttämistä tulokkaiden perehdyttämisessä, kun taas \textcite{pham-ym-2017} toteuttivat eksploratiivisen haastattelu- ja kyselytutkimuksen. Aineiston artikkeleissa eniten oli toteutettu tapaustutkimuksia (yhdeksän). Kyselyt ja haastattelut mainittiin viidessä artikkelissa. Kolme tutkimusta luokiteltiin eksploratiiviseksi. Toimintatutkimuksia oli kaksi: \textcite{bjornson-dingsøyr-2005} kehittivät tutkimuksessaan konsulttiyrityksen mentorointikäytäntöjä toteuttamalla alkuhaastatteluja ja kirjallisuuskatsauksen, joiden perusteella mentorointiohjelma uudistettiin. Immersiivisestä etnografiasta raportoivia artikkeleita aineistossa oli kaksi. \textcite{kumar-ym-2016} raportoivat tutkimuksesta, jossa tutkija työskenteli kohdeyrityksessä ohjelmistokehittäjänä kahdeksan kuukauden ajan käyttäen havainnointia ja haastattelua tutkimusmenetelminä.

\begin{figure}[h]
    \centering
    \includegraphics[width=\textwidth]{media/tutkimustyypit.png}
    \caption{Artikkelien tutkimustyypit}
    \label{kuvio:tutkimustyypit}
\end{figure}

Artikkeleissa mainittiin yhteensä 12 eri tutkimusmenetelmää (\ref{kuvio:tutkimusmenetelmat}). Eniten käytetty oli puolistrukturoitu haastattelu, jota käytettiin yhdentoista artikkelin tutkimuksessa. Kyselyä taas käytettiin seitsemässä: esimerkiksi \textcite{kulkarni-ym-2010} toteuttivat kyselyt vastavalmistuneille ohjelmistokehittäjille ja ohjelmistoalan yritysten HR-asiantuntijoille saadakseen tietoa siitä, minkälaisia perehdytysohjelmia yrityksissä on. Aineistossa havainnointi (5) ja osallistuva havainnointi (2) saivat myös mainintoja. Grounded theory-menetelmä mainittiin neljässä artikkelissa. Esimerkiksi \textcite{viana-ym-2014} tutkivat sitä, miten ohjelmistokehitykseen liittyvä tieto siirtyy kokeneilta osaajilta tulokkaille. 

\begin{figure}[h]
    \centering
    \includegraphics[width=\textwidth]{media/menetelmat.png}
    \caption{Artikkelien tutkimusmenetelmät}
    \label{kuvio:tutkimusmenetelmat}
\end{figure}

Tämän katsauksen tavoitteenahan oli selvittää, minkälaisia perehdytyskäytäntöjä ohjelmistoalan yrityksissä käytetään. Aineiston perusteella voidaan tehdä havaintoja siitä, miten primääritutkimusten tutkimusmenetelmien valinta vaikutti mainittujen perehdytyskäytäntöjen määrään. Kuviossa \ref{kuvio:menetelmilla-havaitut-kaytannot} on esitetty keskiarvot sille, montako perehdytyskäytäntöä keskimäärin mainittiin eri tutkimusmenetelmiä käyttäneissä artikkeleissa. Kuviosta voidaan havaita, että Repertory Grid Technique -menetelmän avulla käytäntöjä löytyi eniten (15). Myös pikaviestinnän (kahdeksan) ja dokumenttien (7) analysointi toi esiin useita käytäntöjä. Myös havainnointi tai osallistuva havainnointi (keskimäärin yht. 9,4) ja haastattelut (8,9) antoivat tietoa käytäntöjen määrästä. Videointi tai näytön videointi taas eivät olleet tehokkaita tapoja löytää käytäntöjä, sillä molemmilla löytyi keskimäärin yksi perehdytyskäytäntö.

\begin{figure}[h]
    \centering
    \includegraphics[width=\textwidth]{media/menetelmilla-havaitut-kaytannot.png}
    \caption{Keskiarvoiset artikkeleissa mainittujen perehdytyskäytäntöjen määrät tutkimusmenetelmittäin}
    \label{kuvio:menetelmilla-havaitut-kaytannot}
\end{figure}

Tutkimusten lähdeaineistojen koko eli datapisteiden määrä vaihteli kolmesta \textcite{kulkarni-ym-2010} 267:ään \textcite{rodeghero-ym-2021}, keskiarvon ollessa 45. Yhteensä katsauksen artikkeissa oli 765 datapistettä eli haastateltua henkilöä, kyselyyn vastaajaa tai havainnoitua perehdytyssessiota.

\subsection{Johtopäätökset}

Katsauksen tulosten perusteella voidaan siis todeta, että Repertory Grid Technique (RGT) -tutkimusmenetelmän avulla havaittiin tehokkaasti käytettyjä perehdytyskäytäntöjä. \textcite{buchan-ym-2019} tutkivat tätä menetelmää käyttämällä sitä, mitkä perehdytyskäytännöt ovat tärkeitä eri perehdytystavoitteiden saavuttamisessa. Tutkimukseen osallistujilta saatiin tietoa kunkin tavoitteen saavuttamiseen hyödynnettävistä käytännöistä sekä näiden välisten yhteyksien vahvuudesta eli siitä, missä määrin käytäntö edistää tavoitteen saavuttamista. Tulosten mukaan esimerkiksi pariohjelmoinnin katsottiin edistävän tuotteen ja kohdealueen tuntemusta. Olemassaolevaan koodiin tutustumisen taas arvioitiin helpottavan ohjelmointi- ja testiautomaatiokäytäntöjen omaksumista. \parencite{buchan-ym-2019}.

\textcite{buchan-ym-2019} toteavat, että RGT:n pääkomponentit ovat (1) tutkimusaihe; (2) elementit, jotka ovat tutkimusaiheen ilmentymiä; (3) konstruktiot, jotka ovat ajatuksia, joita osallistujilla on elementeistä; (4) yhteydet, jotka ovat elementtien ja konstruktioiden välisiä havaittuja suhteita. He ovat  käyttäneet menetelmää niin, että tutkimusaiheena oli ohjelmistokehittäjän perehdyttäminen. Elementteinä olivat perehdyttämisprosessin tavoitteet ja konstrukteina perehdytyskäytännöt. Linkkejä olivat näiden väliset yhteydet. Perehdyttämisprosessin tavoitteet oli määritelty tutkijoiden toimesta etukäteen. Perehdytyskäytännöt selvitettiin tutkimuskohteilta. Tämän menetelmän avulla saatiin siis paitsi tietoa perehdytyskäytännöistä, myös ammattilaisten käsityksiä niiden vaikuttavuudesta. 

RTG-menetelmää voisi hyödyntää niin, että tutkittaisiin tässä kirjallisuuskatsauksessa havaittujen 45 käytännön yhteyksiä perehtymisprosessin tavoitteisiin. Tässä katsauksessa eri käytäntöjen vaikuttavuutta ei arvioitu, joten jatkotutkimusta voitaisiin suunnata siihen. Vaihtoehtoisesti RGT-menetelmän elementteinä voisi olla SWEBOK (Software Engineering Body of Knowledge) -julkaisussa \parencite{swebok} kuvatut ohjelmistokehityksen yleiset osaamisvaatimukset. Tämä voisi auttaa räätälöimään perehdytysprosesseja yksilöllisempään suuntaan riippuen tulokkaan tuen tarpeista ja oppimistavoitteista - onhan kullakin tulokkaalla omat vahvuutensa ja kehittämiskohteensa näiden osaamisvaatimusten alueilla. Perehdytysprosessin yksilöllistämistä suosittelevat myös \textcite{britto-ym-2017} ja \textcite{rodeghero-ym-2021}.


% Johtopäätökset tuloksista: “(muista raportoida havainnot ja johtopäätökset erikseen): pohdinta mitä tämä synteesi tarkoittaa, katsotaan mitä nämä löydökset tarkoittavat aikaisemman kirjallisuuden valossa, mitä ne tuovat siihen lisää ja mitä uusia tutkimusavauksia löydöksistä voi juontua”

\section{Tutkimuskohteiden kontekstit ja kohderyhmät}

Katsauksen tutkimusten yritykset toimivat eri konteksteissa (ks. kuvio \ref{kuvio:kontekstit}). Eniten mainittu konteksti artikkeleissa oli \textit{Agile} eli ketterät menetelmät, joka mainittiin seitsemässä artikkelissa kahdestakymmenestä. Seuraavaksi eniten, viisi mainintaa, saivat globaalisti hajautettu ohjelmistokehitys ja suuret yritykset. Esimerkiksi \textcite{johnson-senges-2010} kertovat Googlen sekä \textcite{rodeghero-ym-2021} \textcite{ju-ym-2021} Microsoftin perehdytyskäytännöistä. Etätyöskentely, pienet yritykset, yrityksen järjestämä akatemia ja keskikokoiset yritykset olivat kontekstina kukin kahdessa artikkelissa. Esimerkiksi \textcite{rodeghero-ym-2021} raportoivat koronapandemian aikana työnsä aloittaneiden ohjelmistokehittäjien perehdyttämisestä.

\begin{figure}[h]
    \centering
    \includegraphics[width=\textwidth]{media/kontekstit.png}
    \caption{Artikkeleissa havaitut kontekstit}
    \label{kuvio:kontekstit}
\end{figure}

Katsauksen artikkelien tutkimuksissa kohderyhmänä olivat useimmiten tulokkaat. Myös kokeneita, junior-ohjelmistokehittäjiä ja esihenkilöitä oli tutkittu, kuten kuviosta (\ref{kuvio:kohderyhmat}) ilmenee. Yhtä lukuun ottamatta kaikissa artikkeleissa kohderyhmänä oli ohjelmistokehittäjät yleisesti, olivat he sitten tulokkaita, vastavalmistuneita, junioreja tai kokeneita. Näiden lisäksi yhdessä artikkelissa kohderyhmänä olivat loppuvaiheen opiskelijat.

\begin{figure}[h]
    \centering
    \includegraphics[width=\textwidth]{media/kohderyhmat.png}
    \caption{Artikkeleissa havaitut kohderyhmät}
    \label{kuvio:kohderyhmat}
\end{figure}


\subsection{Johtopäätökset}

Mitä katsauksen artikkelien tutkimusten konteksteista voidaan sitten päätellä? Ketterät menetelmät, globaalisti hajautettu ohjelmistokehitys tai etätyöskentely oli siis mainittu monissa, yhteensä kahdessatoista artikkelissa. Nämä ovat toki yleisiä ohjelmistokehityksen suuntauksia, mutta näissä konteksteissa myös hyvä perehdytys on erityisen tärkeää. \textcite{gregory-ym-2020} mukaan ketterän kehityksen konteksti aiheuttaa perehtymiselle haasteita, kun tulokkaan on omaksuttava myös ketterät toimintatavat ja ajattelutavat. 

\textcite{britto-ym-2017} toteavat, että etäällä työskentelevien ohjelmistokehittäjien perehdyttäminen olisi yritysten haasteista peräti suurin erityisesti silloin, jos yrityksessä käytetään ketteriä menetelmiä. \textcite{britto-ym-2017} havaitsivat, että ketterät työskentelytavat merkitsivät sitä, että yrityksissä korostetaan enemmän verkostoitumista kollegoiden kanssa kuin edistymisen, työn tulosten ja työskentelytapojen dokumentointia. Juuri kattava ja ajantasainen dokumentaatio olisi kuitenkin tärkeä perehtymisresurssi erityisesti etänä työskenteleville tulokkaille. \textcite{moe-ym-2020} taas mainitsevat, että ketterien menetelmien käyttämisestä koitui tarve jatkuvalle työryhmän sisäiselle kommunikoinnille ja yhteistoiminnalle. Globaalisti hajautetuissa työryhmissä sekä maantieteellinen etäisyys että käytännön seikat, kuten aikaero, aiheuttavat haasteita tälle.

Katsauksen tutkimuksissa kohderyhmänä oli kuudessa artikkelissa junior-ohjelmistokehittäjät tai vastavalmistuneet. Kun tiedetään, että uudet ohjelmistokehittäjät kohtaavat ensimmäisissä työpaikoissaan monia haasteita (ks. luku \ref{luku-tulokkaiden-haasteet}, on hyvä että katsauksessa informantteina oli myös näitä työntekijöitä. Kuviossa \ref{kuvio:ulottuvuudet_kohderyhmat_juniorit} on esitelty sosialisaatioresurssien teorian ulottuvuudet, joihin liittyviä käytäntöjä näissä kuudessa artikkelissa havaittiin. Vertailun vuoksi kuviossa \ref{kuvio:ulottuvuudet_kohderyhmat_kokeneet_tai_hallinto} on vastaavat tiedot artikkeleissa, joissa kohderyhminä olivat esihenkilöt, projektipäälliköt, koulutusohjelmien johtajat, HR-asiantuntijat tai kokeneet ohjelmistokehittäjät. Oleellisin ero näissä osajoukoissa on se, että työn suunnittelun ulottuvuuteen liittyviä käytäntöjä ei mainittu lainkaan niissä artikkeleissa, joissa kohderyhminä oli juniorit tai vastavalmistuneet. Työn tekemisen resurssien ulottuvuuden käytäntöjä mainittiin vain yhdessä artikkelissa. Molemmat ulottuvuudet saivat viisi mainintaa muilta kohderyhmiltä. Olisinkin hyvä tutkia tarkemmin, miten junior-ohjelmistokehittäjien työskentelyä suunnitellaan joko perehdyttävän organisaation tai kehittäjien itsensä toimesta. Myös työn tekemisen resurssien ulottuvuuteen liittyviä käytäntöjä olisi syytä selvittää. Tämän katsauksen artikkeleista \textcite{yates-ym-2020} olivat valinneet näkökulmakseen "information-push":in, eli heitä kiinnosti se, miten organisaation aloitteesta tulokkaille siirretään tietoa. Työn tekemisen resurssien näkökulmasta voisi tutkia sitä, minkälaisin aloittein tulokkaat hyödyntävät esimerkiksi organisaation sisäistä dokumentaatiota tai olemassaolevaa koodia suoriutuakseen työtehtävistään. \textcite{gruman-ym-2006} tekemä tutkimus yliopisto-opiskelijoiden työharjoittelussa ilmenneiden perehdytyskäytäntöjen, tulokkaiden minäpystyvyyden, oma-aloitteisuuden ja sosialisaatiotulosten välisistä suhteista esittelee vahvan viitekehyksen, jonka kautta aihetta voisi tutkia myös sovelluskehityksen alalla.

\begin{figure}[h]
    \centering
    \includegraphics[width=\textwidth]{media/ulottuvuudet_kohderyhmat_juniorit-tai_vastavalmistuneet.png}
    \caption{Artikkeleissa havaitut sosialisaatioresurssien ulottuvuudet, kun kohderyhmänä juniorit tai vastavalmistuneet}
    \label{kuvio:ulottuvuudet_kohderyhmat_juniorit}
\end{figure}

\begin{figure}[h]
    \centering
    \includegraphics[width=\textwidth]{media/ulottuvuudet_kohderyhmat_kokeneet_tai_hallinto.png}
    \caption{Artikkeleissa havaitut sosialisaatioresurssien ulottuvuudet, kun kohderyhmänä kokeneet ohjelmistokehittäjät tai hallinnon työntekijät}
    \label{kuvio:ulottuvuudet_kohderyhmat_kokeneet_tai_hallinto}
\end{figure}

Tämän katsauksen artikkeleissa raportoitiin tutkimuksista erikokoisissa organisaatioissa. \textcite{viana-ym-2014} tutkivat 10 henkilön yritystä, \textcite{johnson-senges-2010} Googlen käytäntöjä ja \textcite{britto-ym-2020} Ericsson AB:ta. Laaja skaala erikokoisia organisaatioita vahvistaa katsauksen tulosten uskottavuutta.

Jäikö sitten jokin oleellinen konteksti tai kohderyhmä vaille havaintoja? Katsauksessa ei ollut mukana artikkeleita, joissa olisi tutkittu julkisen sektorin organisaatioiden käyttämiä perehdytyskäytäntöjä. Vahvasti säänneltyjen alojen kuten lääkinnällisiin laitteisiin, autoteollisuuteen tai maanpuolustukseen liittyvän ohjelmistokehityksen konteksteissa esiintyvistä perehdytyskäytännöistä ei saatu tietoa. Myöskään startup-yrityksiä ei artikkeleissa mainittu. Kohderyhmistä taas mainitsematta jäivät boot camp-tyyppisen intensiivikurssin käyneet ohjelmistokehittäjät. Näistä voidaan juontaa mahdollisia jatkotutkimusaiheita. \textcite{lyon-green-2021} toteavat, että startup-yrityksillä on käytettävissä vain vähän resursseja kouluttaa uusia työntekijöitä,  ja ne pyrkivät siten rekrytoimaan työntekijöitä, joilla on välittömästi sovellettavissa olevat taidot ja jotka pystyvät aloittamaan työskentelyn heti, mikä saattaa joissakin tapauksissa koskea intensiivikurssilla opiskelleita. Näiden kontekstien yhdistelmissä sovellettavat perehtymiskäytännöt olisivat erinomaisen mielenkiintoinen tutkimuskohde, semminkin kun startup-yrityksissä usein toimitaan ketterin menetelmin, mikä jo todetusti aiheuttaa perehtymiselle haasteita.

Yksi katsauksen kiinnostavista löydöksistä oli yhdessä artikkelissa \parencite{kulkarni-ym-2010} mainittu työyhteisötaitokoulutus (engl. \textit{behavioral training}), jota järjestettiin lähes 80\%:ssa tutkituista yrityksistä. Tutkimuksen kyselyyn oli vastannut 32 intialaista IT-alan yritystä, joista kolmannes oli suuria (yli 10 000 työntekijää). Muodollisessa koulutuksessa vastavalmistuneille ohjelmistokehittäjille opetettiin tiimityöskentelyä, suullista ja kirjallista viestintää, johtamistaitoja sekä yritysmaailmassa ja asiakkaiden kanssa käyttäytymisen taitoja. Koulutus kesti keskimäärin 6-7 päivää. \parencite{kulkarni-ym-2010}. Tällaisen koulutuksen järjestäminen näyttää olevan siis tutkituissa yrityksissä varsin yleinen käytäntö. Löydös herättää pohdintoja siitä, onko tarpeen tutkia tarkemmin sitä, minkälaista perehdytyksen tulisi olla monikulttuurisissa tiimeissä, jotta kaikki tiimin jäsenet ymmärtävät toistensa odotukset, toimintatavat ja kulttuuriset seikat. 

\section{Tulosten hyödyntämisen mahdollisuudet}

Luvussa \ref{luku-tulokkaiden-haasteet} kuvailtiin ohjelmistokehittäjien ensitaipaleellaan kohtaamia haasteita viidellä aihealueella, jotka ovat kommunikaatio, yhteistoiminta, tekninen taito, kognitio ja orientoituminen. Miten tämän kirjallisuuskatsauksen tulokset vastaavat näihin haasteisiin? Miten tuloksia voisi hyödyntää ohjelmistokehittäjien perehdytyksessä?

\textcite{begel-simon-2008} toteavat, että kommunikointiin liittyvät haasteet näkyvät tulokkaiden vaikeuksina kirjallisessa, suullisessa ja asiakkaiden kanssa viestiessä. Sosialisaatioresurssien teorian muodollisen työnopetukseen ulottuvuuteen tässä katsauksessa sijoitettu työyhteisötaitokoulutus voisi auttaa tulokkaita viestimisessä. Koulutuksessa voisi opettaa tulokkaille organisaation viestintäkäytännöt ja sovitut asiat. \textcite{begel-simon-2008} mainitsevat myös tulokkaiden vaikeudet pyytää apua tai kysyä tarkentavia kysymyksiä oikea-aikaisesti. Tässä voisi hyödyntää mentoroinnin mahdollisuuksia. Myös yhteistoiminnallinen ohjelmointi parin tai ryhmän kanssa voisi auttaa tulokkaita saamaan oikea-aikaista apua.

\textcite{begel-simon-2008} kertovat, että tulokkaiden on haastavaa työskennellä yhteistoiminnallisesti suurissa tiimeissä tai useiden tiimien kanssa ja että assertiivisuuden puute aiheuttaa tulokkaille pulmia. Tämän katsauksen tuloksista erityisesti tulokkaiden keskinäisen oppimisen ulottuvuutta voisi hyödyntää näihin haasteisiin vastaamiseksi. Jos organisaatiossa aloittaa useita tulokkaita samaan aikaan, he voisivat toimia vertaisryhmänä ja oppia yhdessä. Myös yhteistoiminnallista ohjelmointia voisi hyödyntää - vertaisryhmässä siihen voisi olla matalampi kynnys kun kokeneiden kollegoiden kanssa. Sosialisaatioresurssien teorian ulottuvuuteen 3 (oma-aloitteisuuteen kannustaminen) liitetty "Tulokas haastattelee kokeneita"-käytäntö voisi olla myös yksi vaihtoehto. Voisiko tulokkaan oppiminen olla mielekästä ja relevanttia, jos hän hankkii tietoa ja kysyy kysymyksiä verrattuna siihen, että tulokkaalle järjestetään perehdytyssessioita, joissa aktiivisena toimijana on perehdyttäjä? Tämän katsauksen artikkeleistahan \textcite{yates-ym-2020} olivat valinneet näkökulmakseen "information-push":in, eli heitä kiinnosti se, miten organisaation aloitteesta tulokkaille siirretään tietoa. Voisiko "information-pull" auttaa tulokasta jäsentämään tietoa mielekkäällä tavalla?

Teknisiin taitoihin liittyvät haasteet liittyvät versionhallinnan käytön hallitsemiseen, ohjelmistojen testaamiseen ja ohjelmistokoodissa navigointiin \parencite{begel-simon-2008}. Perehdyttämisen keinoin näitä voisi helpottaa erityisesti ulottuvuuden 11 (muodollinen työnopetus) avulla. Versionhallintaan liittyvä kurssi ja koulutusjakso, jossa tulokkaalle opetettaisiin koodipohjassa navigointia olisivat täsmäkeinot näihin pulmiin. Myös ulottuvuuden 13 (informaatio) arkkitehtuurin ja olemassaolevan koodin esittelyyn liittyvät käytännöt voisivat auttaa. 

\textcite{begel-simon-2008} toteavat, että kognitioon liittyvät haasteet näkyvät siinä, että perehdytys toteutuu työn ohessa vailla struktuuria, tarpeen mukaan tapahtuva perehdyttäminen johtaa siihen, että tulokkaiden osaaminen on jäsentymätöntä. Ratkaisuna voisi olla muodollinen koulutusjakso, jonka avulla varmistettaisiin, että tulokkaat saavat koherentin pohjan osaamiselleen. Myös työn suunnittelun ulottuvuuteen sijoitettu tavoitteiden asettamisen ja arvioinnin käytäntö voisi auttaa jäsentämään opittavia sisältöjä. Ylipäätään perehdyttämistä voisi strukturoida ja suunnitella, kuten myös \textcite{britto-ym-2020} sekä \textcite{hemphill-begel-2011} suosittelevat. Ulottuvuuden 12 (työtehtävät ja työn luonne) viikkoraporttikäytäntö voisi auttaa tulokasta jäsentämään oppimaansa. Sen avulla voisi myös saada tiedoa tulokkaan kokemuksista.

Orientoitumiseen liittyviä haasteita vaikuttaa aiheuttavan liian vähäinen tiedon määrä \parencite{begel-simon-2008}. Ratkaisu on siis itsestäänselvästi sosialisaatioresurssiteorian informaation ulottuvuus. Sopivia käytäntöjä voivat olla olemassaolevan koodin ja arkkitehtuurin esittely. Myös työtehtävien ja työn luonteen ulottuvuuden käytännöistä löytyy kelpo vaihtoehtoja. Työtehtävän kontekstualisoiti ja tulokkaan tehtäväksi etukäteen valmisteltu projekti voisivat auttaa. Orientoitumista tukevat toki myös hyvät työn tekemisen resurssit (ulottuvuus 9), erityisesti ajantasainen ja ymmärrettävä sisäinen dokumentaatio sekä määritellyt ohjelmointikäytännöt.

\chapter{Päätäntä}

TODO metateksti

\section{Yhteenveto}

Tämän systemaattisen kirjallisuuskatsauksen tavoitteena oli selvittää, minkälaisia perehdytyskäytäntöjä ohjelmistoalan yrityksissä käytetään. Katsauksen mahdollisesti soveltuvia artikkeleita löytyi kolmesta eri tietokannasta yhteensä 587 kappaletta. Tutkimusprotokollassa määriteltyjen valintakriteerien soveltamisen jälkeen jäljelle jäi 16 artikkelia. Ainestoa täydennettiin snowballing-kierroksen avulla ja lopullinen artikkelien määrä katsauksessa oli 20.

Tiedonkeruuvaiheessa artikkeleista kerättiin tutkimusprotokollassa määritellyt tiedot, kuten tutkimuksissa käytetyt tutkimustyypit, menetelmät, tutkimuskohteiden kontekstit, tutkimusten kohderyhmät ja tutkimuksissa artikkelissa mainitut yritysten käyttämät perehdytyskäytännöt. Valtaosa tutkimuksista oli tapaus-, kysely- tai haastattelututkimuksia. Käytetyimpiä tutkimusmenetelmiä taas olivat puolistrukturoidut haastattelut. kysely ja havainnointi. Tutkimuskohteina olleissa organisaatioissa toimittiin usein ketterien menetelmien tai globaalisti hajautetun ohjelmistokehityksen  konteksteissa. Kohteina oli niin suuria, pieniä kuin keskisuuriakin yrityksiä. Valtaosassa tutkimuksista kohderyhmänä oli tulokkaat, mutta myös kokeneita ohjelmistokehittäjiä, esihenkilöitä ja muita työntekijöitä oli hyödynnetty informantteina. 

Tässä katsauksessa artikkeleista löydettiin yhteensä 45 erilaista perehdytyskäytäntöä. Käytäntöjä jaoteltiin sosialisaatioresurssien teorian seitsemäntoista ulottuvuuden mukaisesti. Jaottelun myötä havaittiin, että useimmin mainitut perehdytyskäytännöt liittyivät mentorointiin, tulokkaille annettavien työtehtäviin ja niiden luonteeseen sekä tulokkaan saamaan palautteeseen esimerkiksi koodin katselmoinnin muodossa. Näiden kolmen ulottuvuuden jälkeen seuraavaksi eniten korostui orientaatiovaiheen jälkeinen tulokkaiden keskinäinen oppiminen, joka tässä katsauksessa määriteltiin kahdeksanneksitoista, sosialisaatioresurssien teoriaa täydentäväksi ulottuvuudeksi. Tähän ulottuvuuteen liittyviä käytäntöjä mainittiin peräti yhdeksässä artikkelissa 20:sta.


\section{Tuloksiin liittyviä rajoitteita}
\label{luku-rajoitteet}

Katsauksen aineiston hyväksymis- ja hylkäyskriteerien soveltaminen tehtiin vain tutkielman laatijan toimesta, joten henkilökohtainen vinouma on saattanut vaikuttaa siihen, tuliko artikkeli valituksi katsaukseen vai ei. Lisäksi katsauksen tiedonkeruuvaihe toteutettiin vain tutkijan tutkielman laatijan toimesta, joten siinäkin saattaa esiintyä vinoumaa. Molempien vinoumien vaikutuksia on pyritty vähentämään laatimalla selkeät ja yksiselitteiset valintakriteerit sekä selkeästi strukturoitu tiedonkeruusuunnitelma. joita molempia myös pilotointiin. Tutkimusprotokolla myös hyväksytettiin tutkielman ohjaajalla etukäteen.

Aineiston analyysivaiheessa havaittiin, että kerätyissä tiedoissa oli joitakin päällekkäisyyksiä. Esimerkiksi haastattelututkimuksissa haastattelu oli merkitty sekä tutkimuksen tyypiksi että tutkimusmenetelmäksi. Pilottivaihe sisälsi valintakriteerien soveltamisen sekä tiedonkeruuvaiheen. Sen perusteella tiedonkeruustrategiaa tarkennettiin. Pilottivaihe ei kuitenkaan sisältänyt aineiston synteesi- tai analysointivaihetta, joten puutteita ei silloin havaittu. Pilotti olisi ollut hyvä laajentaa koskemaan tiedonkeruun lisäksi myös sen analysointia ja visualisointia, mikä olisi voinut johtaa laadukkaampaan tiedonkeruustrategiaan.

Ovatko katsauksen tulokset sitten kattavia ja yleistettävissä? \textcite{eskola-suoranta-1998} toteavat, että tutkimusaineistoa voidaan arvioida olevan riittävästi silloin, kun aineiston laajentaminen ei enää tuottaisi tutkimuskysymysten kannalta uutta oleellista tietoa. Kyse on aineiston kyllääntymisestä eli saturaatiosta \parencite{eskola-suoranta-1998}. Tämän katsauksen aineiston perusteella vaikuttaa siltä, että aineisto olisi saavuttanut riittävän kyllääntymisen. Artikkelien tiedonkeruun aikana samat perehdytyskäytännöt alkoivat toistua. Tutkielman laatijan arvion mukaan vaikuttaa siltä, että uusien artikkelien lisääminen aineistoon ei olisi tuottanut merkittävästi lisää tietoa. Joitakin yksittäisiä, vain harvoissa organisaatioissa käytössä olevia menetelmiä olisi toki voinut löytyä, mutta kokonaisuuden kannalta sellaiset löydökset olisivat olleet vähäarvoisia. 

Katsauksen tuloksia olisi mahdollista validoida esimerkiksi toteuttamalla ohjelmistoalan ammattilaisille kysely, jonka avulla voitaisiin varmistaa, onko joitakin oleellisia käytäntöjä jäänyt huomaamatta. Pro gradu-tutkielman puitteissa validointiin ei ryhdytty. Yksi tuloksiin liittyvistä rajoitteista on se, että kaikki artikkelit eivät olleet ilmestyneet tietojenkäsittelytieteen alan journal-lehdissä, vaan osa niistä oli julkaistu tieteellisten konferenssien kokoomateoksessa (engl. \textit{proceedings}). Näiden artikkeleiden jättäminen katsauksen ulkopuolelle olisi kuitenkin vähentänyt aineiston määrää niin pieneksi (6 kpl), että tulosten luotettavuus vaarantuisi.

Rajoituksena voidaan mainita myös se, että aivan kaikissa katsauksen artikkeleissa tutkimuksen nimenomaisena tavoitteena ei ollut selvittää perehdytyskäytäntöjen käyttöä. Kaikissa artikkeleissa kyllä mainittiin näitä käytäntöjä, mutta artikkelien näkökulmissa oli variaatiota. Esimerkiksi \textcite{yates-ym-2020} olivat kiinnostuneita nimenomaan ohjelman ymmärtämiseen (engl. \textit{program comprehension}) liittyvästä perehdyttämisestä sekä \textcite{hemphill-begel-2011} perehdyttämisvaiheen haasteista virtuaalisissa tiimeissä. On siis mahdollista, että näissä artikkeleissa ei ole mainittu kaikkia organisaatioiden käytäntöjä, mikä puolestaan heijastuu myös katsauksen tuloksiin. Katsauksessa oli kuitenkin useita artikkeleita, joissa tutkittiin juuri perehdytyskäytäntöjä, mikä voidaan lukea katsauksen eduksi
%
\parencites%
    {ju-ym-2021}%
    {britto-ym-2020}%
    {moe-ym-2020}%
    {viviani-murphy-2019}%
    {buchan-ym-2019}%
    {johnson-senges-2010}%
\relax.
%

\section{Tutkimusprotokollan arviointi}
\label{luku-tutkimusprotokollan-arviointi}

\textcite{kitchenham-charters-2007} toteavat, että kirjallisuuskatsausten sisäistä koheesiota voidaan arvioida kolmesta eri näkökulmasta: ensinnäkin hakulausekkeiden tulee olla johdettu asianmukaisesti tutkimuskysymyksistä. Toiseksi artikkeleista tulee kerätä tiedot, jotka vastaavat tutkimuskysymyksiin. Lisäksi aineiston analysoinnin tulee olla asianmukainen tutkimuskysymyksiin vastaamiseksi. Seuraavassa arvioidaan tämän katsauksen tutkimusprotokollaa näiden kriteerien perusteella.

Tämän katsauksen tutkimuskysymyksenä oli: \textit{“Minkälaisia käytäntöjä ohjelmistokehitystä tekevissä organisaatioissa käytetään ohjelmistokehittäjien perehdyttämiseksi työhönsä?”}. Tiedonhaussa käytetyt hakulausekkeet johdettiin tutkimuskysymyksestä luvussa \ref{luku:tutkimuskysymys} esitellyn PICOC-struktuurin avulla. Hakulauseketta muodostettaessa tehtiin useita pilottihakuja, joiden tulosten perusteella hakulauseketta tarkennettiin. Hakulausekkeen asianmukaisuutta arvioitiin katsauksen valmistuttua tekemällä Google Scholar-tietokantaan summittaisia testihakuja mm. hakulausekkeella “onboarding developers” ja silmäilemällä muutamia kymmeniä hakutuloksia. Tämän suuntaa-antavan arvioinnin perusteella vaikutti siltä, että oleelliset artikkelit oli onnistuttu sisällyttämään katsaukseen mukaan, sillä hakutuloksista ei löytynyt artikkeleita, jotka joko eivät olisi mukana katsauksessa tai jotka olisi katsauksen valintakriteerien perusteella hylätty. 

Tässä katsauksessa tavoitteena oli selvittää ohjelmistokehittäjien perehdytyskäytäntöjä. Katsauksen artikkeleista kerättiin tiedonkeruuvaiheessa kaikki niissä mainitut käytännöt, joita jäsennettiin sosialisaatioresurssien teorian avulla. Voidaan arvioida, että katsauksessa on kerätty asianmukaiset tiedot tutkimuskysymyksiin vastaamiseksi. Myös aineiston analyysimenetelmä vaikuttaa onnistuneelta, sillä katsauksessa löydettiin monipuolisesti eri sosialisaatioresurssien teorian ulottuvuuksiin liittyviä käytäntöjä.

Tutkimusprotokollaa voidaan arvioida myös sen perusteella, miksi snowballing-kierroksella katsaukseen lisätyt neljä artikkelia eivät löytyneet hakuprotokollassa määritellyillä hakulausekkeilla %
\parencites%
    {bjornson-dingsøyr-2005}%
    {kulkarni-ym-2010}%
    {hemphill-begel-2011}%
    {viviani-murphy-2019}%
\relax.
%
Tämän voidaan arvioida johtuvan osittain siitä, että perehdyttämiskäytäntöihin liittyvässä tutkimuksessa käytettävä käsitteistö on paikoin vakiintumatonta. Kuten taulukosta \ref{tbl:picoc-ulottuvuudet} ilmenee, uusiin työntekijöihin voidaan viitata sanoilla \textit{entry level, novice, junior, newcomer, new hire, apprentice} tai \textit{new team member}. Termeissä on toki myös merkityseroja. \textcite{kulkarni-ym-2010} käyttävät käsitettä \textit{fresh graduates} ja \textcite{bjornson-dingsøyr-2005} käsitettä \textit{less experienced employees}. \textcite{begel-simon-2008} taas puhuvat NSD:istä (eng. \textit{new software developers}). Käsitteistön vakiintumattomuus tulee ottaa huomioon mahdollisessa jatkotutkimuksessa. Yksi artikkeleista \parencite{hemphill-begel-2011} on ilmestynyt Microsoft Technical Report -julkaisusarjassa, mikä selittää sen, ettei sitä löytynyt tiedonhaussa käytetyistä tietokannoista (ACM, IEEE ja SCOPUS). Neljäs snowballing-kierroksella lisätty artikkeli (\textcite{viviani-murphy-2019} oli mukana hakutuloksissa, mutta siihen oli tutkimusten valintaprosessissa virheellisesti sovellettu hylkäyskriteeriä "muu aihe". Tämä havainto antaa aihetta arvioida, onko muitakin artikkeleita jäänyt aineiston ulkopuolelle heikoin perustein. Kuten \textcite{kitchenham-charters-2007} toteavat, tutkielman laatija voi arvioida artikkelien valintakriteerien soveltamista joko keskustelemalla hyväksytyistä ja hylätyistä artikkeleista ohjaajan tai muun asiantuntijan kanssa tai soveltamalla test-retest-menetelmää, jossa satunnaisotos aineistosta valitaan arvioitavaksi uudelleen. Tässä katsauksessa sovellettiin jälkimmäistä menetelmää, jossa katsauksen hylätyistä artikkeleista (567 kpl) kaksikymmentä valittiin arvioitavaksi uudelleen. Näissä artikkeleissa hylkäyskriteerejä oli sovellettu asianmukaisesti. Tämän satunnaisotoksen perusteella virheellisesti hylätty artikkeli \parencite{viviani-murphy-2019} olisi ollut yksittäistapaus. 


\section{Eettiset ulottuvuudet}

Tämän systemaattisen kirjallisuuskatsauksen eettisiä ulottuvuuksia voidaan arvioida eri näkökulmista. Ensinnäkin voidaan tarkastella katsaukseen sisällytettyjä tutkimuksia. Lisäksi voidaan arvioida katsausta itseään.

Tämän katsauksen kahdestakymmenestä artikkelista eettisiä näkökulmia ei ekspliittisesti käsitelty yhdessäkään artikkelissa. \textcite{moe-ym-2020} ovat hyödyntäneet aineistonaan mm. Slack-pikaviestintälogeja eli työntekijöiden kirjoittamia viestejä, mutta eettisiä seikkoja ei artikkelissa problematisoida lainkaan. \textcite{yates-ym-2020} kyllä toteavat, että heidän videoimiensa perehdytyssessioiden osallistujilta saatiin kirjalliset suostumukset. Heidän käyttämässään kysylylomakkeessa on lisäksi mainittu, että vastaaja saa olla vastaamatta kysymyksiin tai vetäytyä tutkimuksesta milloin vain. 

Seitsemässä katsauksen artikkelissa arvioitiin tutkimuksen validiteettia. Useimmat raportoidut validiteetin haasteet liittyvät siihen, että tutkimus on toteutettu vain yhdessä yrityksessä (esim. \textcite{johnson-senges-2010}), joten tulokset eivät välttämättä ole yleistettävissä laajemmin. \textcite{viana-ym-2014} taas ovat haastatelleet vain kolme työntekijää, mikä tulee ottaa huomioon arvioitaessa tutkimuksen validiteettia. \textcite{hemphill-begel-2011} työskentelivät tutkimassaan yrityksessä esihenkilönä ja työntekijänä. Myös \textcite{kumar-wallace-2019} kertovat artikkelissaan toisen tutkijoista työskennelleen tutkitussa yrityksessä. Molemmissa artikkeleissa pidättäydytään arvioimasta sitä, miten asetelma on mahdollisesti vaikuttanut tutkimukseen, sen tuloksiin ja niiden yleistettävyyteen.

Voidaan siis todeta, että eettisiä näkökulmia ei ole artikkeleissa arvioitu seikkaperäisesti. Myös validiteettia on arvioitu niukasti.

Kirjallisuuskatsauksissa artikkelien valinta vaikuttaa tutkimuksen laatuun ja tutkimustuloksiin, joten on tärkeää että lähdeiden valinta perustuu objektiivisiin kriteereihin. Tämä pyrittiin katsausta tehtäessä varmistamaan laatimalla tutkimusprotokolla huolellisesti ja hyväksyttämällä se ohjaajalla ennen tiedonhakujen tekemistä. Katsauksen data on saatavilla internetissä osoitteessa 
 \href{https://github.com/sarlijes/gradu-2023}{https://github.com/sarlijes/gradu-2023}. Saatavilla on bibtex-tiedosto, joka sisältää tiedot kaikista artikkeleista, jotka löytyivät tiedonhaussa. Lisäksi tutkimuksen tiedonkeruuvaiheessa muodostunut tieto on saatavilla .csv-tiedostona. Se sisältää kaiken bibtex-tietojen lisäksi katsaukseen hyväksyttyjen artikkelien osalta niistä kerätyt tiedot ja hylättyjen artikkelien osalta hylkäämissyyn. 

Tämän katsauksen raportoinnissa on pyritty noudattamaan tarkkuutta ja objektiivisuutta. Tuloksiin liittyviä rajoitteita on arvioitu luvussa \ref{luku-rajoitteet}. Tutkimusprotokollaa taas on arvioitu luvussa \ref{luku-tutkimusprotokollan-arviointi}.


\printbibliography
\end{document}